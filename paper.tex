\documentclass{llncs}
\usepackage{amsmath}
\usepackage{amssymb}
\usepackage{tikz}
\usepackage{graphicx,color}
\usepackage{multicol}
\usepackage{float}

\usepackage{isabelle,isabellesym}
\usepackage[greek,english]{babel}
\usepackage{pdfsetup}

\isabellestyle{it}

\newfloat{schemes}{ptb}{sch}

\usetikzlibrary{arrows,shapes,calc}

\newcommand{\nkat}{\overline{\phantom{x}}}
\newcommand{\llbrace}{\hspace{1pt} \lbrace \kern -3pt \vert \hspace{1pt}}
\newcommand{\rrbrace}{\hspace{0.5pt} \rbrace \kern -5pt \vert \hspace{3pt}}
\newcommand{\ifthenelse}[3]{\text{IF $#1$ THEN $#2$ ELSE $#3$}}
\newcommand{\while}[2]{\text{WHILE $#1$ DO $#2$ WEND}}
\newcommand{\triple}[3]{\llbrace #1 \rrbrace #2 \llbrace #3 \rrbrace}

\begin{document}

\tikzstyle{block} = [rectangle,thick,draw=black,minimum height=6mm]
\tikzstyle{line} = [draw,thick, -latex']
\tikzstyle{test} = [diamond,thick,draw=black]
\tikzstyle{false} = [above right of=t2, node distance=2.5mm]
\tikzstyle{true} = [above left of=t2, node distance=2.5mm]

\title{Schematic KA Paper}

\author{Alasdair Armstrong\inst{1} \and Georg Struth\inst{1} \and Tjark Weber\inst{2}}

\institute{Department of Computer Science, University of Sheffield, UK,\\
\email{$\{$a.armstrong,g.struth$\}$@dcs.shef.ac.uk}
\and
Department of Information Technology, Uppsala University, Sweden,\\
\email{tjark.weber@it.uu.se}}

\maketitle

\begin{abstract}
  FIXME: In this paper we describe a formalisation of Dexter Kozen's
  schematic Kleene algebra in Isabelle/HOL. In addition, we will
  demonstrate how this approach can be extended to support Hoare logic
  style verification of programs, and how in this setting Isabelle can
  be used to fully automate the verification of simple algorithms. I
  will finally discuss promising directions for future work in this
  area, such as adding support for reasoning with concurrent programs.
\end{abstract}

\section{Introduction}

\newpage
...
\pagebreak

\section{Kleene Algebra and Kleene Algebra with Tests}
\label{sec:alg}

A dioid is a structure $(D, +, \cdot, 0, 1)$ where $(D, +, 0)$ is a
semilattice with a least element $0$, $(D, \cdot, 1)$ is a monoid,
multiplication distributes over addition and $0\cdot x = 0 = x\cdot
0$. The natural partial order, $x \le y$ is defined as $x + y = y$. A
Kleene algebra~\cite{Kozen} is a dioid extended with an additional
unary operation $^\ast$, which additionally satisifies the four axioms
\begin{align*}
1 + xx^\ast &\le x^\ast,\\
1 + x^\ast x &\le x^\ast,\\
z + xy \le y &\to x^\ast z \le y,\\
z + yx \le y &\to zx^\ast \le y.
\end{align*}
Intuitively, one can think of the operations of Kleene algebra having
the following meanings: $+$ represents non-deterministic choice
between two actions, while $\cdot$ represents sequential
composition. $1$ is the skip action, and $0$ is the abortive
action. The star operation represents finite iteration, i.e. $x^\ast$
represents zero or more executions of the $x$ action.

In order to model actual programming constructs such as if statements
and while loops we require some notion of a test or assertion. To
facilitate this, we use Kozen's Kleene algebra with
tests~\cite{Kozen97}. Formally, a Kleene algebra with tests (KAT) is
a two sorted structure $(K,B,+,\cdot,^\star,0,1,\nkat)$ where
$(B,+,\cdot,\nkat,0,1)$ is a boolean algebra and
$(K,+,\cdot,^\ast,0,1)$ is a Kleene algebra such that the dioid
$(B,+,\cdot,0,1)$ is a subalgebra of $(K,+,\cdot,0,1)$. The $\nkat$
operator is only defined for elements in $B$. As per Kozen, we use
$p,q,r,\dots$ for arbitrary elements of $K$ and $a,b,c,\dots$ for
tests.

If statements and while loops can be encoded in KAT in the following way:
\begin{align*}
\ifthenelse{b}{p}{q} &\overset{\text{def}}{=} bp + \overline{b}q,\\
\while{b}{p} &\overset{\text{def}}{=} (bp)^\ast \overline{b}.
\end{align*}
It is also sometimes convenient to write sequential composition as $p;
q$ rather than $pq$, where $;$ has a lower precedence than $\cdot$
does. In KAT the $+$, $\cdot$, $0$ and $1$ operators now take on two
meanings. For arbitrary elements of $K$, they have the intuitive
meanings described above, but for tests, they also represent
disjunction, conjunction, false and true respectively. These two
separate usages of the operators do not conflict in any way.

In KAT, Hoare triples can be defined as follows:
\begin{align}
\label{eq:kathoare}
\triple{b}{p}{c} \overset{\text{def}}{\Longleftrightarrow} bp = bpc \Longleftrightarrow bp\overline{c} = 0.
\end{align}
Intuitively, $bp = bpc$ states that if $p$ is executed in a state
satisfying $b$, then $c$ must hold in the resulting state, while
$bp\overline{c} = 0$ states that there are no actions $p$ such that
when executed in a state satisfying $b$ will result in a state
satisfying $c$. All the rules of Hoare logic except the assignment
rules can be derivied in KAT. This fragment is known as propositional
Hoare logic (PHL). KAT, in fact is strictly more powerful than PHL. I
supports the deriviation of new Hoare-style inference rules and
reasoning about program equivalences. KAT also permits an elegant
equational style of reasoning, as it does not rely on any specialised
syntactic constructs, unlike Hoare logic or dynamic logic.

\section{Schematic KAT and Flowchart Schemes}
\label{sec:skat}

The absence of the assignment rule is main obstacle towards verifying
and analysing programs with KAT. To overcome this, axioms for
assignment have been added. These are targeted mainly at semantics of
flowchart schemes, hence the name \emph{schematic KAT}
(SKAT)~\cite{Angus01}. For an overview of flowchart schemes,
transformation rules between such schemes and the general problem of
scheme equivalence, see Manna's book~\cite{Manna}. We formalise
Kozen's approach to flowchart equivalence in SKAT, which in in turn is
inspired by Manna's work. In this section we describe the conceptual
development of SKAT together with its formalisation in Isabelle.

A \emph{ranked alphabet} or signature $\Sigma$ consists of a family of
function symbols $f,g,\dots$ and relation symbols $P,Q,\dots$ together
with an arity with maps symbols to natural numbers. In Isabelle we
have represented ranked alphabets as a type class. Variables are
simply represented by natural numbers. A subset of these are designated
as output variables.

Terms are defined as a datatype in Isabelle in the obvious way.

\begin{isabellebody}
\isanewline
\isacommand{datatype}\isamarkupfalse%
\ {\isaliteral{27}{\isacharprime}}a\ trm\ {\isaliteral{3D}{\isacharequal}}\ App\ {\isaliteral{27}{\isacharprime}}a\ {\isaliteral{22}{\isachardoublequoteopen}}{\isaliteral{27}{\isacharprime}}a\ trm\ list{\isaliteral{22}{\isachardoublequoteclose}}\ {\isaliteral{7C}{\isacharbar}}\ Var\ nat\isanewline
\end{isabellebody}

In the definition there are no arity checks. We omit them for
pragmatic reasons in order to avoid overloading proofs with additional
obligations. In practice verifications will simply fail if arity
constraints are violated.

Predicates are needed to formalise tests, they are again implemented
as a simple datatype.

\begin{isabellebody}
\isanewline
\isacommand{datatype}\isamarkupfalse%
\ {\isaliteral{27}{\isacharprime}}a\ pred\ {\isaliteral{3D}{\isacharequal}}\ Pred\ {\isaliteral{27}{\isacharprime}}a\ {\isaliteral{22}{\isachardoublequoteopen}}{\isaliteral{27}{\isacharprime}}a\ trm\ list{\isaliteral{22}{\isachardoublequoteclose}}\isanewline
\end{isabellebody}

Using terms and predicates we can now define the expressions of
schematic Kleene algebra.

\begin{isabellebody}
\isanewline
\isacommand{datatype}\isamarkupfalse%
\ {\isaliteral{27}{\isacharprime}}a\ skat{\isaliteral{5F}{\isacharunderscore}}expr\ {\isaliteral{3D}{\isacharequal}}\isanewline
\ \ \ \ SKLeaf\ nat\ {\isaliteral{22}{\isachardoublequoteopen}}{\isaliteral{27}{\isacharprime}}a\ trm{\isaliteral{22}{\isachardoublequoteclose}}\isanewline
\ \ {\isaliteral{7C}{\isacharbar}}\ SKPlus\ {\isaliteral{22}{\isachardoublequoteopen}}{\isaliteral{27}{\isacharprime}}a\ skat{\isaliteral{5F}{\isacharunderscore}}expr{\isaliteral{22}{\isachardoublequoteclose}}\ {\isaliteral{22}{\isachardoublequoteopen}}{\isaliteral{27}{\isacharprime}}a\ skat{\isaliteral{5F}{\isacharunderscore}}expr{\isaliteral{22}{\isachardoublequoteclose}}\ {\isaliteral{28}{\isacharparenleft}}\isakeyword{infixl}\ {\isaliteral{22}{\isachardoublequoteopen}}{\isaliteral{3A}{\isacharcolon}}{\isaliteral{5C3C6F706C75733E}{\isasymoplus}}{\isaliteral{3A}{\isacharcolon}}{\isaliteral{22}{\isachardoublequoteclose}}\ {\isadigit{7}}{\isadigit{0}}{\isaliteral{29}{\isacharparenright}}\isanewline
\ \ {\isaliteral{7C}{\isacharbar}}\ SKMult\ {\isaliteral{22}{\isachardoublequoteopen}}{\isaliteral{27}{\isacharprime}}a\ skat{\isaliteral{5F}{\isacharunderscore}}expr{\isaliteral{22}{\isachardoublequoteclose}}\ {\isaliteral{22}{\isachardoublequoteopen}}{\isaliteral{27}{\isacharprime}}a\ skat{\isaliteral{5F}{\isacharunderscore}}expr{\isaliteral{22}{\isachardoublequoteclose}}\ {\isaliteral{28}{\isacharparenleft}}\isakeyword{infixl}\ {\isaliteral{22}{\isachardoublequoteopen}}{\isaliteral{3A}{\isacharcolon}}{\isaliteral{5C3C6F646F743E}{\isasymodot}}{\isaliteral{3A}{\isacharcolon}}{\isaliteral{22}{\isachardoublequoteclose}}\ {\isadigit{8}}{\isadigit{0}}{\isaliteral{29}{\isacharparenright}}\isanewline
\ \ {\isaliteral{7C}{\isacharbar}}\ SKStar\ {\isaliteral{22}{\isachardoublequoteopen}}{\isaliteral{27}{\isacharprime}}a\ skat{\isaliteral{5F}{\isacharunderscore}}expr{\isaliteral{22}{\isachardoublequoteclose}}\isanewline
\ \ {\isaliteral{7C}{\isacharbar}}\ SKBool\ {\isaliteral{22}{\isachardoublequoteopen}}{\isaliteral{27}{\isacharprime}}a\ pred\ bexpr{\isaliteral{22}{\isachardoublequoteclose}}\isanewline
\ \ {\isaliteral{7C}{\isacharbar}}\ SKOne\isanewline
\ \ {\isaliteral{7C}{\isacharbar}}\ SKZero\isanewline
\end{isabellebody}

In this datatype, SKLeaf is the constructor for assignments, since it
is built from a variable term pair. The other constructors provide
syntax the programming constructs of sequential composition,
conditionals and while loops as in KAT. The type \textit{'a pred
  bexpr} represents boolean combinations of predicates, which are the
tests in SKAT. The connection between this syntax and the graphical
language of flowchart schemes is discussed in Kozen's paper.

Using this syntax we can now define a notion of flowchart
equivalence. We use Isabelle's quotient types for this purpose. First
we define the obvious congruence on SKAT terms that includes the KAT
axioms and the following assignment axioms
\begin{align*}
x := s; y := t &= y := t[x/s]; x := s &(y \in FV(s))\\
x := s; y := t &= x := s; y := t[x/s] &(x \in FV(s))\\
x := s; x := t &= x := t[x/s]\\
\varphi[x/t]; x := t &= x := t; \varphi
\end{align*}
For brevity, we only show the equivalence rules and a single Kleene
algebra axiom in Isabelle. Additional recursive functions for free
variables and substitutions support the assignment axioms.

\begin{isabellebody}
\isanewline
\isacommand{inductive}\isamarkupfalse%
\ skat{\isaliteral{5F}{\isacharunderscore}}con\ {\isaliteral{3A}{\isacharcolon}}{\isaliteral{3A}{\isacharcolon}}\ {\isaliteral{22}{\isachardoublequoteopen}}{\isaliteral{27}{\isacharprime}}a{\isaliteral{3A}{\isacharcolon}}{\isaliteral{3A}{\isacharcolon}}ranked{\isaliteral{5F}{\isacharunderscore}}alphabet\ skat{\isaliteral{5F}{\isacharunderscore}}expr\ {\isaliteral{5C3C52696768746172726F773E}{\isasymRightarrow}}\ {\isaliteral{27}{\isacharprime}}a\ skat{\isaliteral{5F}{\isacharunderscore}}expr\ {\isaliteral{5C3C52696768746172726F773E}{\isasymRightarrow}}\ bool{\isaliteral{22}{\isachardoublequoteclose}}\isanewline
\ \ \isakeyword{where}\isanewline
\ \ refl\ {\isaliteral{5B}{\isacharbrackleft}}intro{\isaliteral{5D}{\isacharbrackright}}{\isaliteral{3A}{\isacharcolon}}\ {\isaliteral{22}{\isachardoublequoteopen}}skat{\isaliteral{5F}{\isacharunderscore}}con\ x\ x{\isaliteral{22}{\isachardoublequoteclose}}\isanewline
{\isaliteral{7C}{\isacharbar}}\ sym\ {\isaliteral{5B}{\isacharbrackleft}}sym{\isaliteral{5D}{\isacharbrackright}}{\isaliteral{3A}{\isacharcolon}}\ {\isaliteral{22}{\isachardoublequoteopen}}skat{\isaliteral{5F}{\isacharunderscore}}con\ x\ y\ {\isaliteral{5C3C4C6F6E6772696768746172726F773E}{\isasymLongrightarrow}}\ skat{\isaliteral{5F}{\isacharunderscore}}con\ y\ x{\isaliteral{22}{\isachardoublequoteclose}}\isanewline
{\isaliteral{7C}{\isacharbar}}\ trans\ {\isaliteral{5B}{\isacharbrackleft}}trans{\isaliteral{5D}{\isacharbrackright}}{\isaliteral{3A}{\isacharcolon}}\ {\isaliteral{22}{\isachardoublequoteopen}}skat{\isaliteral{5F}{\isacharunderscore}}con\ x\ y\ {\isaliteral{5C3C4C6F6E6772696768746172726F773E}{\isasymLongrightarrow}}\ skat{\isaliteral{5F}{\isacharunderscore}}con\ y\ z\ {\isaliteral{5C3C4C6F6E6772696768746172726F773E}{\isasymLongrightarrow}}\ skat{\isaliteral{5F}{\isacharunderscore}}con\ x\ z{\isaliteral{22}{\isachardoublequoteclose}}\isanewline
$\vdots$ \isanewline
{\isaliteral{7C}{\isacharbar}}\ mult{\isaliteral{5F}{\isacharunderscore}}assoc{\isaliteral{3A}{\isacharcolon}}\ {\isaliteral{22}{\isachardoublequoteopen}}skat{\isaliteral{5F}{\isacharunderscore}}con\ {\isaliteral{28}{\isacharparenleft}}SKMult\ {\isaliteral{28}{\isacharparenleft}}SKMult\ x\ y{\isaliteral{29}{\isacharparenright}}\ z{\isaliteral{29}{\isacharparenright}}\ {\isaliteral{28}{\isacharparenleft}}SKMult\ x\ {\isaliteral{28}{\isacharparenleft}}SKMult\ y\ z{\isaliteral{29}{\isacharparenright}}{\isaliteral{29}{\isacharparenright}}{\isaliteral{22}{\isachardoublequoteclose}}\isanewline
$\vdots$\isanewline
\end{isabellebody}

In Isabelle we can now formally take the quotient of SKAT expressions
with respect to this congruence.

\begin{isabellebody}
\isanewline
\isacommand{quotient{\isaliteral{5F}{\isacharunderscore}}type}\isamarkupfalse%
\ {\isaliteral{27}{\isacharprime}}a\ skat\ {\isaliteral{3D}{\isacharequal}}\ {\isaliteral{22}{\isachardoublequoteopen}}{\isaliteral{27}{\isacharprime}}a{\isaliteral{3A}{\isacharcolon}}{\isaliteral{3A}{\isacharcolon}}ranked{\isaliteral{5F}{\isacharunderscore}}alphabet\ skat{\isaliteral{5F}{\isacharunderscore}}expr{\isaliteral{22}{\isachardoublequoteclose}}\ {\isaliteral{2F}{\isacharslash}}\ skat{\isaliteral{5F}{\isacharunderscore}}con\isanewline
\end{isabellebody}

Once a notion of equivalence on SKAT expressions is available we can
define additional syntactic sugar by lifting constructors to SKAT operations.

\begin{isabellebody}
\isanewline
\isacommand{lift{\isaliteral{5F}{\isacharunderscore}}definition}\isamarkupfalse%
\ skat{\isaliteral{5F}{\isacharunderscore}}plus\ {\isaliteral{3A}{\isacharcolon}}{\isaliteral{3A}{\isacharcolon}}\ {\isaliteral{22}{\isachardoublequoteopen}}{\isaliteral{27}{\isacharprime}}a{\isaliteral{3A}{\isacharcolon}}{\isaliteral{3A}{\isacharcolon}}ranked{\isaliteral{5F}{\isacharunderscore}}alphabet\ skat\ {\isaliteral{5C3C52696768746172726F773E}{\isasymRightarrow}}\ {\isaliteral{27}{\isacharprime}}a\ skat\ {\isaliteral{5C3C52696768746172726F773E}{\isasymRightarrow}}\ {\isaliteral{27}{\isacharprime}}a\ skat{\isaliteral{22}{\isachardoublequoteclose}}
\isanewline\ \ {\isaliteral{28}{\isacharparenleft}}\isakeyword{infixl}\ {\isaliteral{22}{\isachardoublequoteopen}}{\isaliteral{2B}{\isacharplus}}{\isaliteral{22}{\isachardoublequoteclose}}\ {\isadigit{7}}{\isadigit{0}}{\isaliteral{29}{\isacharparenright}}\ \isakeyword{is}\ SKPlus\isanewline
%
\isadelimproof
\ \ %
\endisadelimproof
%
\isatagproof
\isacommand{by}\isamarkupfalse%
\ {\isaliteral{28}{\isacharparenleft}}rule\ add{\isaliteral{5F}{\isacharunderscore}}compat{\isaliteral{2C}{\isacharcomma}}\ assumption{\isaliteral{2B}{\isacharplus}}{\isaliteral{29}{\isacharparenright}}%
\endisatagproof\isanewline
\end{isabellebody}

We have used Isabelle's transfer tactic to provide nice notation and lift definitions from the congruence.

\begin{isabellebody}
\isanewline
\isacommand{lemma}\isamarkupfalse%
\ skat{\isaliteral{5F}{\isacharunderscore}}assign{\isadigit{1}}{\isaliteral{3A}{\isacharcolon}}\ {\isaliteral{22}{\isachardoublequoteopen}}{\isaliteral{5C3C6C6272616B6B3E}{\isasymlbrakk}}x\ {\isaliteral{5C3C6E6F7465713E}{\isasymnoteq}}\ y{\isaliteral{3B}{\isacharsemicolon}}\ y\ {\isaliteral{5C3C6E6F74696E3E}{\isasymnotin}}\ FV\ s{\isaliteral{5C3C726272616B6B3E}{\isasymrbrakk}}\ {\isaliteral{5C3C4C6F6E6772696768746172726F773E}{\isasymLongrightarrow}}\ {\isaliteral{28}{\isacharparenleft}}x\ {\isaliteral{3A}{\isacharcolon}}{\isaliteral{3D}{\isacharequal}}\ s\ {\isaliteral{5C3C63646F743E}{\isasymcdot}}\ y\ {\isaliteral{3A}{\isacharcolon}}{\isaliteral{3D}{\isacharequal}}\ t{\isaliteral{29}{\isacharparenright}}\ {\isaliteral{3D}{\isacharequal}}\ {\isaliteral{28}{\isacharparenleft}}y\ {\isaliteral{3A}{\isacharcolon}}{\isaliteral{3D}{\isacharequal}}\ t{\isaliteral{5B}{\isacharbrackleft}}x{\isaliteral{7C}{\isacharbar}}s{\isaliteral{5D}{\isacharbrackright}}\ {\isaliteral{5C3C63646F743E}{\isasymcdot}}\ x\ {\isaliteral{3A}{\isacharcolon}}{\isaliteral{3D}{\isacharequal}}\ s{\isaliteral{29}{\isacharparenright}}{\isaliteral{22}{\isachardoublequoteclose}}\isanewline
%
\isadelimproof
\ \ %
\endisadelimproof
%
\isatagproof
\isacommand{by}\isamarkupfalse%
\ {\isaliteral{28}{\isacharparenleft}}transfer{\isaliteral{2C}{\isacharcomma}}\ rule\ skat{\isaliteral{5F}{\isacharunderscore}}con{\isaliteral{2E}{\isachardot}}assign{\isadigit{1}}{\isaliteral{29}{\isacharparenright}}%
\endisatagproof\isanewline
\end{isabellebody}

We can now formally prove via an interpretation statement that the quotient algebra constructed forms a KAT.

\begin{isabellebody}
\isanewline
\isacommand{definition}\isamarkupfalse%
\ tests\ {\isaliteral{3A}{\isacharcolon}}{\isaliteral{3A}{\isacharcolon}}\ {\isaliteral{22}{\isachardoublequoteopen}}{\isaliteral{27}{\isacharprime}}a{\isaliteral{3A}{\isacharcolon}}{\isaliteral{3A}{\isacharcolon}}ranked{\isaliteral{5F}{\isacharunderscore}}alphabet\ skat\ ord{\isaliteral{22}{\isachardoublequoteclose}}\ \isakeyword{where}\isanewline
\ \ {\isaliteral{22}{\isachardoublequoteopen}}tests\ {\isaliteral{5C3C65717569763E}{\isasymequiv}}\ {\isaliteral{5C3C6C706172723E}{\isasymlparr}}carrier\ {\isaliteral{3D}{\isacharequal}}\ test{\isaliteral{5F}{\isacharunderscore}}set{\isaliteral{2C}{\isacharcomma}}\ le\ {\isaliteral{3D}{\isacharequal}}\ {\isaliteral{28}{\isacharparenleft}}{\isaliteral{5C3C6C616D6264613E}{\isasymlambda}}x\ y{\isaliteral{2E}{\isachardot}}\ skat{\isaliteral{5F}{\isacharunderscore}}plus\ x\ y\ {\isaliteral{3D}{\isacharequal}}\ y{\isaliteral{29}{\isacharparenright}}{\isaliteral{5C3C72706172723E}{\isasymrparr}}{\isaliteral{22}{\isachardoublequoteclose}}\isanewline
\end{isabellebody}

\begin{isabellebody}
\isacommand{definition}\isamarkupfalse%
\ free{\isaliteral{5F}{\isacharunderscore}}kat\ {\isaliteral{3A}{\isacharcolon}}{\isaliteral{3A}{\isacharcolon}}\ {\isaliteral{22}{\isachardoublequoteopen}}{\isaliteral{27}{\isacharprime}}a{\isaliteral{3A}{\isacharcolon}}{\isaliteral{3A}{\isacharcolon}}ranked{\isaliteral{5F}{\isacharunderscore}}alphabet\ skat\ test{\isaliteral{5F}{\isacharunderscore}}algebra{\isaliteral{22}{\isachardoublequoteclose}}\ \isakeyword{where}\isanewline
\ \ {\isaliteral{22}{\isachardoublequoteopen}}free{\isaliteral{5F}{\isacharunderscore}}kat\ {\isaliteral{3D}{\isacharequal}}\ {\isaliteral{5C3C6C706172723E}{\isasymlparr}}carrier\ {\isaliteral{3D}{\isacharequal}}\ UNIV{\isaliteral{2C}{\isacharcomma}}\ plus\ {\isaliteral{3D}{\isacharequal}}\ skat{\isaliteral{5F}{\isacharunderscore}}plus{\isaliteral{2C}{\isacharcomma}}\ mult\ {\isaliteral{3D}{\isacharequal}}\ skat{\isaliteral{5F}{\isacharunderscore}}mult{\isaliteral{2C}{\isacharcomma}}\ one\ {\isaliteral{3D}{\isacharequal}}\ skat{\isaliteral{5F}{\isacharunderscore}}one{\isaliteral{2C}{\isacharcomma}}\isanewline
\ \ \ \ \ \ \ \ \ \ \ \ \ \ \ \ \ zero\ {\isaliteral{3D}{\isacharequal}}\ skat{\isaliteral{5F}{\isacharunderscore}}zero{\isaliteral{2C}{\isacharcomma}}\ star\ {\isaliteral{3D}{\isacharequal}}\ skat{\isaliteral{5F}{\isacharunderscore}}star{\isaliteral{2C}{\isacharcomma}}\ test{\isaliteral{5F}{\isacharunderscore}}algebra{\isaliteral{2E}{\isachardot}}test\ {\isaliteral{3D}{\isacharequal}}\ tests{\isaliteral{5C3C72706172723E}{\isasymrparr}}{\isaliteral{22}{\isachardoublequoteclose}}\isanewline
\end{isabellebody}

\begin{isabellebody}
\isacommand{interpretation}\isamarkupfalse%
\ skt{\isaliteral{3A}{\isacharcolon}}\ kat\ free{\isaliteral{5F}{\isacharunderscore}}kat\isanewline
\ \ $\dots$\isanewline
\end{isabellebody}

Proving this statement requires a substantial amount of work. First,
we have implemented Kleene algebra with tests (with carrier sets) in
Isabelle. Second, we have shown that the quotient algebra constructed
satisfies the KAT axioms, including the axioms of boolean algebra for
the test subalgebra. A main complication arises from the fact that
boolean complementation is only defined on the subalgebra; as such it
cannot be directly lifted from the congruence and must be defined
indirectly using the indefinite description operator.

When defining a quotient type, Isabelle automatically generates two
coercion functions for us. The \textit{abs-skat} function maps
elements of type \textit{'a skat-expr} to elements of the quotient
algebra type \textit{'a skat}, while the \textit{rep-skat} function
maps in the converse direction. Both these functions are based on
Isabelle's definite description operator, which can be
unwieldy. However, as our types are inductively defined we can as well
use the following recursive function instead of \textit{abs-skat}.

\begin{isabellebody}
\isanewline
\isacommand{primrec}\isamarkupfalse%
\ skat{\isaliteral{5F}{\isacharunderscore}}unfold\ {\isaliteral{3A}{\isacharcolon}}{\isaliteral{3A}{\isacharcolon}}\ {\isaliteral{22}{\isachardoublequoteopen}}{\isaliteral{27}{\isacharprime}}a{\isaliteral{3A}{\isacharcolon}}{\isaliteral{3A}{\isacharcolon}}ranked{\isaliteral{5F}{\isacharunderscore}}alphabet\ skat{\isaliteral{5F}{\isacharunderscore}}expr\ {\isaliteral{5C3C52696768746172726F773E}{\isasymRightarrow}}\ {\isaliteral{27}{\isacharprime}}a\ skat{\isaliteral{22}{\isachardoublequoteclose}}\ {\isaliteral{28}{\isacharparenleft}}{\isaliteral{22}{\isachardoublequoteopen}}{\isaliteral{5C3C6C666C6F6F723E}{\isasymlfloor}}{\isaliteral{5F}{\isacharunderscore}}{\isaliteral{5C3C72666C6F6F723E}{\isasymrfloor}}{\isaliteral{22}{\isachardoublequoteclose}}\ {\isaliteral{5B}{\isacharbrackleft}}{\isadigit{1}}{\isadigit{1}}{\isadigit{1}}{\isaliteral{5D}{\isacharbrackright}}\ {\isadigit{1}}{\isadigit{1}}{\isadigit{0}}{\isaliteral{29}{\isacharparenright}}\ \isakeyword{where}\isanewline
\ \ {\isaliteral{22}{\isachardoublequoteopen}}skat{\isaliteral{5F}{\isacharunderscore}}unfold\ {\isaliteral{28}{\isacharparenleft}}SKLeaf\ x\ y{\isaliteral{29}{\isacharparenright}}\ {\isaliteral{3D}{\isacharequal}}\ x\ {\isaliteral{3A}{\isacharcolon}}{\isaliteral{3D}{\isacharequal}}\ y{\isaliteral{22}{\isachardoublequoteclose}}\isanewline
{\isaliteral{7C}{\isacharbar}}\ {\isaliteral{22}{\isachardoublequoteopen}}skat{\isaliteral{5F}{\isacharunderscore}}unfold\ {\isaliteral{28}{\isacharparenleft}}SKPlus\ x\ y{\isaliteral{29}{\isacharparenright}}\ {\isaliteral{3D}{\isacharequal}}\ skat{\isaliteral{5F}{\isacharunderscore}}unfold\ x\ {\isaliteral{2B}{\isacharplus}}\ skat{\isaliteral{5F}{\isacharunderscore}}unfold\ y{\isaliteral{22}{\isachardoublequoteclose}}\isanewline
{\isaliteral{7C}{\isacharbar}}\ {\isaliteral{22}{\isachardoublequoteopen}}skat{\isaliteral{5F}{\isacharunderscore}}unfold\ {\isaliteral{28}{\isacharparenleft}}SKMult\ x\ y{\isaliteral{29}{\isacharparenright}}\ {\isaliteral{3D}{\isacharequal}}\ skat{\isaliteral{5F}{\isacharunderscore}}unfold\ x\ {\isaliteral{5C3C63646F743E}{\isasymcdot}}\ skat{\isaliteral{5F}{\isacharunderscore}}unfold\ y{\isaliteral{22}{\isachardoublequoteclose}}\isanewline
{\isaliteral{7C}{\isacharbar}}\ {\isaliteral{22}{\isachardoublequoteopen}}skat{\isaliteral{5F}{\isacharunderscore}}unfold\ {\isaliteral{28}{\isacharparenleft}}SKBool\ p{\isaliteral{29}{\isacharparenright}}\ {\isaliteral{3D}{\isacharequal}}\ test{\isaliteral{5F}{\isacharunderscore}}unfold\ p{\isaliteral{22}{\isachardoublequoteclose}}\isanewline
{\isaliteral{7C}{\isacharbar}}\ {\isaliteral{22}{\isachardoublequoteopen}}skat{\isaliteral{5F}{\isacharunderscore}}unfold\ SKOne\ {\isaliteral{3D}{\isacharequal}}\ {\isaliteral{5C3C6F6E653E}{\isasymone}}{\isaliteral{22}{\isachardoublequoteclose}}\isanewline
{\isaliteral{7C}{\isacharbar}}\ {\isaliteral{22}{\isachardoublequoteopen}}skat{\isaliteral{5F}{\isacharunderscore}}unfold\ SKZero\ {\isaliteral{3D}{\isacharequal}}\ {\isaliteral{5C3C7A65726F3E}{\isasymzero}}{\isaliteral{22}{\isachardoublequoteclose}}\isanewline
{\isaliteral{7C}{\isacharbar}}\ {\isaliteral{22}{\isachardoublequoteopen}}skat{\isaliteral{5F}{\isacharunderscore}}unfold\ {\isaliteral{28}{\isacharparenleft}}SKStar\ x{\isaliteral{29}{\isacharparenright}}\ {\isaliteral{3D}{\isacharequal}}\ {\isaliteral{28}{\isacharparenleft}}skat{\isaliteral{5F}{\isacharunderscore}}unfold\ x{\isaliteral{29}{\isacharparenright}}\isaliteral{5C3C5E7375703E}{}\isactrlsup {\isaliteral{5C3C737461723E}{\isasymstar}}{\isaliteral{22}{\isachardoublequoteclose}}\isanewline
\end{isabellebody}

Mathematically, \textit{skat-unfold} is a homomorphism. It is very
useful when defining tactics.

\section{Formalising a Metatheorem}
\label{sec:metatheorem}

Now we formalise a metatheorem (Lemma 4.4 in~\cite{Angus01}) that is
useful for various purposes, including checking commutativity
conditions, elimination of useless variables and renaming
variables. In Isabelle these tasks are captured by tactics which
instantiate the metatheorem.

\begin{isabellebody}
\isanewline
\isacommand{theorem}\isamarkupfalse%
\ metatheorem{\isaliteral{3A}{\isacharcolon}}\isanewline
\ \ \isakeyword{assumes}\ f{\isaliteral{5F}{\isacharunderscore}}homo{\isaliteral{3A}{\isacharcolon}}\ {\isaliteral{22}{\isachardoublequoteopen}}kat{\isaliteral{5F}{\isacharunderscore}}homomorphism\ f{\isaliteral{22}{\isachardoublequoteclose}}\isanewline
\ \ \isakeyword{and}\ g{\isaliteral{5F}{\isacharunderscore}}homo{\isaliteral{3A}{\isacharcolon}}\ {\isaliteral{22}{\isachardoublequoteopen}}kat{\isaliteral{5F}{\isacharunderscore}}homomorphism\ g{\isaliteral{22}{\isachardoublequoteclose}}\isanewline
\ \ \isakeyword{and}\ atomic{\isaliteral{3A}{\isacharcolon}}\ {\isaliteral{22}{\isachardoublequoteopen}}{\isaliteral{5C3C416E643E}{\isasymAnd}}a{\isaliteral{2E}{\isachardot}}\ a\ {\isaliteral{5C3C696E3E}{\isasymin}}\ atoms\ p\ {\isaliteral{5C3C4C6F6E6772696768746172726F773E}{\isasymLongrightarrow}}\ f\ a\ {\isaliteral{5C3C63646F743E}{\isasymcdot}}\ x\ {\isaliteral{3D}{\isacharequal}}\ x\ {\isaliteral{5C3C63646F743E}{\isasymcdot}}\ g\ a{\isaliteral{22}{\isachardoublequoteclose}}\isanewline
\ \ \isakeyword{shows}\ {\isaliteral{22}{\isachardoublequoteopen}}f\ p\ {\isaliteral{5C3C63646F743E}{\isasymcdot}}\ x\ {\isaliteral{3D}{\isacharequal}}\ x\ {\isaliteral{5C3C63646F743E}{\isasymcdot}}\ g\ p{\isaliteral{22}{\isachardoublequoteclose}}\isanewline
\end{isabellebody}

The proof is by induction on $p$ and follows essentially that of
Kozen. The statement of the theorem uses the predicate
\textit{kat-homomorphism} which states that $f$ and $g$ are morphisms
of KAT. This notion is defined in Isabelle as a locale with the
obvious conditions. The functions $f$ and $g$ map from the SKAT term
algebra into SKAT, hence they have the same type as
\textit{skat-unfold}.

We have instantiated the metatheorem as follows:

\begin{isabellebody}
\isanewline
\isacommand{lemmas}\isamarkupfalse%
\ skat{\isaliteral{5F}{\isacharunderscore}}comm\ {\isaliteral{3D}{\isacharequal}}\ metatheorem{\isaliteral{5B}{\isacharbrackleft}}OF\ skat{\isaliteral{5F}{\isacharunderscore}}unfold{\isaliteral{5F}{\isacharunderscore}}homo\ skat{\isaliteral{5F}{\isacharunderscore}}unfold{\isaliteral{5F}{\isacharunderscore}}homo{\isaliteral{5D}{\isacharbrackright}}\isanewline
\end{isabellebody}

\begin{isabellebody}
\isacommand{lemma}\isamarkupfalse%
\ eliminate{\isaliteral{5F}{\isacharunderscore}}variables{\isaliteral{3A}{\isacharcolon}}\isanewline
\ \ \isakeyword{assumes}\ no{\isaliteral{5F}{\isacharunderscore}}reads{\isaliteral{3A}{\isacharcolon}}\ {\isaliteral{22}{\isachardoublequoteopen}}x\ {\isaliteral{5C3C6E6F74696E3E}{\isasymnotin}}\ reads\ p{\isaliteral{22}{\isachardoublequoteclose}}\isanewline
\ \ \isakeyword{shows}\ {\isaliteral{22}{\isachardoublequoteopen}}{\isaliteral{5C3C6C666C6F6F723E}{\isasymlfloor}}p{\isaliteral{5C3C72666C6F6F723E}{\isasymrfloor}}\ {\isaliteral{5C3C63646F743E}{\isasymcdot}}\ x\ {\isaliteral{3A}{\isacharcolon}}{\isaliteral{3D}{\isacharequal}}\ null\ {\isaliteral{3D}{\isacharequal}}\ {\isaliteral{5C3C6C666C6F6F723E}{\isasymlfloor}}eliminate\ x\ p{\isaliteral{5C3C72666C6F6F723E}{\isasymrfloor}}\ {\isaliteral{5C3C63646F743E}{\isasymcdot}}\ x\ {\isaliteral{3A}{\isacharcolon}}{\isaliteral{3D}{\isacharequal}}\ null{\isaliteral{22}{\isachardoublequoteclose}}\isanewline
\ \ $\dots$\isanewline
\end{isabellebody}

Lemma \textit{skat-comm} states that if $x$ commutes with all atomic
expressions of $p$, then $x$ commutes with $p$; an observation due to
Kozen. This has been proved by simply instantiating $f$ and $g$ in the
metatheorem using a proof that \textit{skat-unfold} is a KAT
homomorphism. Lemma \textit{eliminate-variables} is Kozen's Lemma
$4.5$ which states that if the variable $x$ is not read in $p$, then
setting it to null will eliminate it from $p$. In the statement of the
lemma, \textit{reads $p$} is a recursive function which returns all
the variables on the right-hand side of all assignments within $p$,
and the function \textit{eliminate x p} removes all assignments to $x$
in $p$.

We have used these lemmas to program tactics that check for
commutativity and eliminate variables. These tactics take expressions
of the quotient algebra and coerce them into the term algebra to
perform these syntactic manipulations. All the machinery for these
coercions, such as \textit{skat-unfold} is thereby hidden to the user.

\begin{align*} %FIXME
&1 := Var\ 2; 3 := Var\ 4\ =\ 3 := Var\ 2; 1 := Var\ 2\\
&\text{  by skat\_comm}
\end{align*}

\section{Verification of Flowchart Equivalence}

We now use our implementation of SKAT to formally verify a well known
flowchart equivalence example in Isabelle. This example is attributed
by Manna to Paterson. We follow Angus and Kozen's analysis in
SKAT. The two flowcharts can be found at page $16$ and $17$ in Angus
and Kozen's paper or page 254 and 258 in Manna's
book~\cite{Manna}. Manna essentially uses graph transformations on
flowchart diagrams in a rather informal way to prove the equivalence
between these schemes. Angus and Kozen use an alternative algebraic
approach. Their manual proof covers six pages. Our contribution is the
verification of their proof in Isabelle.

The two flowcharts, translated into SKAT, are as follows.

\begin{isabellebody}
\isanewline
\isacommand{definition}\isamarkupfalse%
\ scheme{\isadigit{1}}\ {\isaliteral{3A}{\isacharcolon}}{\isaliteral{3A}{\isacharcolon}}\ {\isaliteral{22}{\isachardoublequoteopen}}kzp\ skat\ list{\isaliteral{22}{\isachardoublequoteclose}}\ \isakeyword{where}\ {\isaliteral{22}{\isachardoublequoteopen}}scheme{\isadigit{1}}\ {\isaliteral{5C3C65717569763E}{\isasymequiv}}\isanewline
\ \ {\isaliteral{5B}{\isacharbrackleft}}\ {\isadigit{1}}\ {\isaliteral{3A}{\isacharcolon}}{\isaliteral{3D}{\isacharequal}}\ vx\isanewline
\ \ {\isaliteral{2C}{\isacharcomma}}\ {\isadigit{4}}\ {\isaliteral{3A}{\isacharcolon}}{\isaliteral{3D}{\isacharequal}}\ f\ {\isaliteral{28}{\isacharparenleft}}Var\ {\isadigit{1}}{\isaliteral{29}{\isacharparenright}}\isanewline
\ \ {\isaliteral{2C}{\isacharcomma}}\ {\isadigit{1}}\ {\isaliteral{3A}{\isacharcolon}}{\isaliteral{3D}{\isacharequal}}\ f\ {\isaliteral{28}{\isacharparenleft}}Var\ {\isadigit{1}}{\isaliteral{29}{\isacharparenright}}\isanewline
\ \ {\isaliteral{2C}{\isacharcomma}}\ {\isadigit{2}}\ {\isaliteral{3A}{\isacharcolon}}{\isaliteral{3D}{\isacharequal}}\ g\ {\isaliteral{28}{\isacharparenleft}}Var\ {\isadigit{1}}{\isaliteral{29}{\isacharparenright}}\ {\isaliteral{28}{\isacharparenleft}}Var\ {\isadigit{4}}{\isaliteral{29}{\isacharparenright}}\isanewline
\ \ {\isaliteral{2C}{\isacharcomma}}\ {\isadigit{3}}\ {\isaliteral{3A}{\isacharcolon}}{\isaliteral{3D}{\isacharequal}}\ g\ {\isaliteral{28}{\isacharparenleft}}Var\ {\isadigit{1}}{\isaliteral{29}{\isacharparenright}}\ {\isaliteral{28}{\isacharparenleft}}Var\ {\isadigit{1}}{\isaliteral{29}{\isacharparenright}}\isanewline
\ \ {\isaliteral{2C}{\isacharcomma}}\ loop\isanewline
\ \ \ \ {\isaliteral{5B}{\isacharbrackleft}}\ {\isaliteral{21}{\isacharbang}}{\isaliteral{28}{\isacharparenleft}}P\ {\isaliteral{28}{\isacharparenleft}}Var\ {\isadigit{1}}{\isaliteral{29}{\isacharparenright}}{\isaliteral{29}{\isacharparenright}}\isanewline
\ \ \ \ {\isaliteral{2C}{\isacharcomma}}\ {\isadigit{1}}\ {\isaliteral{3A}{\isacharcolon}}{\isaliteral{3D}{\isacharequal}}\ f\ {\isaliteral{28}{\isacharparenleft}}Var\ {\isadigit{1}}{\isaliteral{29}{\isacharparenright}}\isanewline
\ \ \ \ {\isaliteral{2C}{\isacharcomma}}\ {\isadigit{2}}\ {\isaliteral{3A}{\isacharcolon}}{\isaliteral{3D}{\isacharequal}}\ g\ {\isaliteral{28}{\isacharparenleft}}Var\ {\isadigit{1}}{\isaliteral{29}{\isacharparenright}}\ {\isaliteral{28}{\isacharparenleft}}Var\ {\isadigit{4}}{\isaliteral{29}{\isacharparenright}}\isanewline
\ \ \ \ {\isaliteral{2C}{\isacharcomma}}\ {\isadigit{3}}\ {\isaliteral{3A}{\isacharcolon}}{\isaliteral{3D}{\isacharequal}}\ g\ {\isaliteral{28}{\isacharparenleft}}Var\ {\isadigit{1}}{\isaliteral{29}{\isacharparenright}}\ {\isaliteral{28}{\isacharparenleft}}Var\ {\isadigit{1}}{\isaliteral{29}{\isacharparenright}}\isanewline
\ \ \ \ {\isaliteral{5D}{\isacharbrackright}}\isanewline
\ \ {\isaliteral{2C}{\isacharcomma}}\ P\ {\isaliteral{28}{\isacharparenleft}}Var\ {\isadigit{1}}{\isaliteral{29}{\isacharparenright}}\isanewline
\ \ {\isaliteral{2C}{\isacharcomma}}\ {\isadigit{1}}\ {\isaliteral{3A}{\isacharcolon}}{\isaliteral{3D}{\isacharequal}}\ f\ {\isaliteral{28}{\isacharparenleft}}Var\ {\isadigit{3}}{\isaliteral{29}{\isacharparenright}}\isanewline
\ \ {\isaliteral{2C}{\isacharcomma}}\ loop\isanewline
\ \ \ \ {\isaliteral{5B}{\isacharbrackleft}}\ {\isaliteral{21}{\isacharbang}}{\isaliteral{28}{\isacharparenleft}}P\ {\isaliteral{28}{\isacharparenleft}}Var\ {\isadigit{4}}{\isaliteral{29}{\isacharparenright}}{\isaliteral{29}{\isacharparenright}}\ {\isaliteral{2B}{\isacharplus}}\ seq\isanewline
\ \ \ \ \ \ {\isaliteral{5B}{\isacharbrackleft}}\ P\ {\isaliteral{28}{\isacharparenleft}}Var\ {\isadigit{4}}{\isaliteral{29}{\isacharparenright}}\isanewline
\ \ \ \ \ \ {\isaliteral{2C}{\isacharcomma}}\ {\isaliteral{28}{\isacharparenleft}}{\isaliteral{21}{\isacharbang}}{\isaliteral{28}{\isacharparenleft}}P\ {\isaliteral{28}{\isacharparenleft}}Var\ {\isadigit{2}}{\isaliteral{29}{\isacharparenright}}{\isaliteral{29}{\isacharparenright}}\ {\isaliteral{5C3C63646F743E}{\isasymcdot}}\ {\isadigit{2}}\ {\isaliteral{3A}{\isacharcolon}}{\isaliteral{3D}{\isacharequal}}\ f\ {\isaliteral{28}{\isacharparenleft}}Var\ {\isadigit{2}}{\isaliteral{29}{\isacharparenright}}{\isaliteral{29}{\isacharparenright}}\isaliteral{5C3C5E7375703E}{}\isactrlsup {\isaliteral{5C3C737461723E}{\isasymstar}}\isanewline
\ \ \ \ \ \ {\isaliteral{2C}{\isacharcomma}}\ P\ {\isaliteral{28}{\isacharparenleft}}Var\ {\isadigit{2}}{\isaliteral{29}{\isacharparenright}}\isanewline
\ \ \ \ \ \ {\isaliteral{2C}{\isacharcomma}}\ {\isaliteral{21}{\isacharbang}}\ {\isaliteral{28}{\isacharparenleft}}P\ {\isaliteral{28}{\isacharparenleft}}Var\ {\isadigit{3}}{\isaliteral{29}{\isacharparenright}}{\isaliteral{29}{\isacharparenright}}\isanewline
\ \ \ \ \ \ {\isaliteral{2C}{\isacharcomma}}\ {\isadigit{4}}\ {\isaliteral{3A}{\isacharcolon}}{\isaliteral{3D}{\isacharequal}}\ f\ {\isaliteral{28}{\isacharparenleft}}Var\ {\isadigit{1}}{\isaliteral{29}{\isacharparenright}}\isanewline
\ \ \ \ \ \ {\isaliteral{2C}{\isacharcomma}}\ {\isadigit{1}}\ {\isaliteral{3A}{\isacharcolon}}{\isaliteral{3D}{\isacharequal}}\ f\ {\isaliteral{28}{\isacharparenleft}}Var\ {\isadigit{1}}{\isaliteral{29}{\isacharparenright}}\isanewline
\ \ \ \ \ \ {\isaliteral{5D}{\isacharbrackright}}\isanewline
\ \ \ \ {\isaliteral{2C}{\isacharcomma}}\ {\isadigit{2}}\ {\isaliteral{3A}{\isacharcolon}}{\isaliteral{3D}{\isacharequal}}\ g\ {\isaliteral{28}{\isacharparenleft}}Var\ {\isadigit{1}}{\isaliteral{29}{\isacharparenright}}\ {\isaliteral{28}{\isacharparenleft}}Var\ {\isadigit{4}}{\isaliteral{29}{\isacharparenright}}\isanewline
\ \ \ \ {\isaliteral{2C}{\isacharcomma}}\ {\isadigit{3}}\ {\isaliteral{3A}{\isacharcolon}}{\isaliteral{3D}{\isacharequal}}\ g\ {\isaliteral{28}{\isacharparenleft}}Var\ {\isadigit{1}}{\isaliteral{29}{\isacharparenright}}\ {\isaliteral{28}{\isacharparenleft}}Var\ {\isadigit{1}}{\isaliteral{29}{\isacharparenright}}\isanewline
\ \ \ \ {\isaliteral{2C}{\isacharcomma}}\ loop\isanewline
\ \ \ \ \ \ {\isaliteral{5B}{\isacharbrackleft}}\ {\isaliteral{21}{\isacharbang}}{\isaliteral{28}{\isacharparenleft}}P\ {\isaliteral{28}{\isacharparenleft}}Var\ {\isadigit{1}}{\isaliteral{29}{\isacharparenright}}{\isaliteral{29}{\isacharparenright}}\isanewline
\ \ \ \ \ \ {\isaliteral{2C}{\isacharcomma}}\ {\isadigit{1}}\ {\isaliteral{3A}{\isacharcolon}}{\isaliteral{3D}{\isacharequal}}\ f\ {\isaliteral{28}{\isacharparenleft}}Var\ {\isadigit{1}}{\isaliteral{29}{\isacharparenright}}\isanewline
\ \ \ \ \ \ {\isaliteral{2C}{\isacharcomma}}\ {\isadigit{2}}\ {\isaliteral{3A}{\isacharcolon}}{\isaliteral{3D}{\isacharequal}}\ g\ {\isaliteral{28}{\isacharparenleft}}Var\ {\isadigit{1}}{\isaliteral{29}{\isacharparenright}}\ {\isaliteral{28}{\isacharparenleft}}Var\ {\isadigit{4}}{\isaliteral{29}{\isacharparenright}}\isanewline
\ \ \ \ \ \ {\isaliteral{2C}{\isacharcomma}}\ {\isadigit{3}}\ {\isaliteral{3A}{\isacharcolon}}{\isaliteral{3D}{\isacharequal}}\ g\ {\isaliteral{28}{\isacharparenleft}}Var\ {\isadigit{1}}{\isaliteral{29}{\isacharparenright}}\ {\isaliteral{28}{\isacharparenleft}}Var\ {\isadigit{1}}{\isaliteral{29}{\isacharparenright}}\isanewline
\ \ \ \ \ \ {\isaliteral{5D}{\isacharbrackright}}\isanewline
\ \ \ \ {\isaliteral{2C}{\isacharcomma}}\ P\ {\isaliteral{28}{\isacharparenleft}}Var\ {\isadigit{1}}{\isaliteral{29}{\isacharparenright}}\isanewline
\ \ \ \ {\isaliteral{2C}{\isacharcomma}}\ {\isadigit{1}}\ {\isaliteral{3A}{\isacharcolon}}{\isaliteral{3D}{\isacharequal}}\ f\ {\isaliteral{28}{\isacharparenleft}}Var\ {\isadigit{3}}{\isaliteral{29}{\isacharparenright}}\isanewline
\ \ \ \ {\isaliteral{5D}{\isacharbrackright}}\isanewline
\ \ {\isaliteral{2C}{\isacharcomma}}\ P\ {\isaliteral{28}{\isacharparenleft}}Var\ {\isadigit{4}}{\isaliteral{29}{\isacharparenright}}\isanewline
\ \ {\isaliteral{2C}{\isacharcomma}}\ {\isaliteral{28}{\isacharparenleft}}{\isaliteral{21}{\isacharbang}}{\isaliteral{28}{\isacharparenleft}}P\ {\isaliteral{28}{\isacharparenleft}}Var\ {\isadigit{2}}{\isaliteral{29}{\isacharparenright}}{\isaliteral{29}{\isacharparenright}}\ {\isaliteral{5C3C63646F743E}{\isasymcdot}}\ {\isadigit{2}}\ {\isaliteral{3A}{\isacharcolon}}{\isaliteral{3D}{\isacharequal}}\ f\ {\isaliteral{28}{\isacharparenleft}}Var\ {\isadigit{2}}{\isaliteral{29}{\isacharparenright}}{\isaliteral{29}{\isacharparenright}}\isaliteral{5C3C5E7375703E}{}\isactrlsup {\isaliteral{5C3C737461723E}{\isasymstar}}\isanewline
\ \ {\isaliteral{2C}{\isacharcomma}}\ P\ {\isaliteral{28}{\isacharparenleft}}Var\ {\isadigit{2}}{\isaliteral{29}{\isacharparenright}}\isanewline
\ \ {\isaliteral{2C}{\isacharcomma}}\ P\ {\isaliteral{28}{\isacharparenleft}}Var\ {\isadigit{3}}{\isaliteral{29}{\isacharparenright}}\isanewline
\ \ {\isaliteral{2C}{\isacharcomma}}\ {\isadigit{0}}\ {\isaliteral{3A}{\isacharcolon}}{\isaliteral{3D}{\isacharequal}}\ Var\ {\isadigit{2}}\isanewline
\ \ {\isaliteral{2C}{\isacharcomma}}\ halt\isanewline
\ \ {\isaliteral{5D}{\isacharbrackright}}{\isaliteral{22}{\isachardoublequoteclose}}\isanewline
\isanewline
\isacommand{definition}\isamarkupfalse%
\ scheme{\isadigit{2}}\ \isakeyword{where}\ {\isaliteral{22}{\isachardoublequoteopen}}scheme{\isadigit{2}}\ {\isaliteral{5C3C65717569763E}{\isasymequiv}}\isanewline
\ \ {\isaliteral{5B}{\isacharbrackleft}}\ {\isadigit{2}}\ {\isaliteral{3A}{\isacharcolon}}{\isaliteral{3D}{\isacharequal}}\ f\ vx\isanewline
\ \ {\isaliteral{2C}{\isacharcomma}}\ P\ {\isaliteral{28}{\isacharparenleft}}Var\ {\isadigit{2}}{\isaliteral{29}{\isacharparenright}}\isanewline
\ \ {\isaliteral{2C}{\isacharcomma}}\ {\isadigit{2}}\ {\isaliteral{3A}{\isacharcolon}}{\isaliteral{3D}{\isacharequal}}\ g\ {\isaliteral{28}{\isacharparenleft}}Var\ {\isadigit{2}}{\isaliteral{29}{\isacharparenright}}\ {\isaliteral{28}{\isacharparenleft}}Var\ {\isadigit{2}}{\isaliteral{29}{\isacharparenright}}\isanewline
\ \ {\isaliteral{2C}{\isacharcomma}}\ skat{\isaliteral{5F}{\isacharunderscore}}star\ {\isaliteral{28}{\isacharparenleft}}seq\isanewline
\ \ \ \ {\isaliteral{5B}{\isacharbrackleft}}\ {\isaliteral{21}{\isacharbang}}{\isaliteral{28}{\isacharparenleft}}P\ {\isaliteral{28}{\isacharparenleft}}Var\ {\isadigit{2}}{\isaliteral{29}{\isacharparenright}}{\isaliteral{29}{\isacharparenright}}\isanewline
\ \ \ \ {\isaliteral{2C}{\isacharcomma}}\ {\isadigit{2}}\ {\isaliteral{3A}{\isacharcolon}}{\isaliteral{3D}{\isacharequal}}\ f\ {\isaliteral{28}{\isacharparenleft}}f\ {\isaliteral{28}{\isacharparenleft}}Var\ {\isadigit{2}}{\isaliteral{29}{\isacharparenright}}{\isaliteral{29}{\isacharparenright}}\isanewline
\ \ \ \ {\isaliteral{2C}{\isacharcomma}}\ P\ {\isaliteral{28}{\isacharparenleft}}Var\ {\isadigit{2}}{\isaliteral{29}{\isacharparenright}}\isanewline
\ \ \ \ {\isaliteral{2C}{\isacharcomma}}\ {\isadigit{2}}\ {\isaliteral{3A}{\isacharcolon}}{\isaliteral{3D}{\isacharequal}}\ g\ {\isaliteral{28}{\isacharparenleft}}Var\ {\isadigit{2}}{\isaliteral{29}{\isacharparenright}}\ {\isaliteral{28}{\isacharparenleft}}Var\ {\isadigit{2}}{\isaliteral{29}{\isacharparenright}}\isanewline
\ \ \ \ {\isaliteral{5D}{\isacharbrackright}}{\isaliteral{29}{\isacharparenright}}\isanewline
\ \ {\isaliteral{2C}{\isacharcomma}}\ P\ {\isaliteral{28}{\isacharparenleft}}Var\ {\isadigit{2}}{\isaliteral{29}{\isacharparenright}}\isanewline
\ \ {\isaliteral{2C}{\isacharcomma}}\ {\isadigit{0}}\ {\isaliteral{3A}{\isacharcolon}}{\isaliteral{3D}{\isacharequal}}\ Var\ {\isadigit{2}}\isanewline
\ \ {\isaliteral{2C}{\isacharcomma}}\ halt\isanewline
\ \ {\isaliteral{5D}{\isacharbrackright}}{\isaliteral{22}{\isachardoublequoteclose}}\isanewline
\isanewline
\end{isabellebody}

In the definition statement \textit{kzp} is an instantiation of the
ranked alphabet type class. In the code, lists delimited by brackets
indicate blocks of sequential code. Loop expressions indicate the star
of a block of code that follows. The \textit{seq} function converts a
block of code into an SKAT expression. The \textit{halt} command sets
all non output variables used in the scheme to null. To make algebraic
reasoning more efficient, we follow Angus and Kozen in introducing
abbreviations for atomic commands and tests. The flowchart equivalence
problem can then be expressed more succinctly in SKAT as follows.

\begin{isabellebody}
\isanewline
seq {\isaliteral{5B}{\isacharbrackleft}}x{\isadigit{1}}{\isaliteral{2C}{\isacharcomma}}p{\isadigit{4}}{\isadigit{1}}{\isaliteral{2C}{\isacharcomma}}p{\isadigit{1}}{\isadigit{1}}{\isaliteral{2C}{\isacharcomma}}q{\isadigit{2}}{\isadigit{1}}{\isadigit{4}}{\isaliteral{2C}{\isacharcomma}}q{\isadigit{3}}{\isadigit{1}}{\isadigit{1}}{\isaliteral{2C}{\isacharcomma}}loop\ {\isaliteral{5B}{\isacharbrackleft}}{\isaliteral{21}{\isacharbang}}a{\isadigit{1}}{\isaliteral{2C}{\isacharcomma}}p{\isadigit{1}}{\isadigit{1}}{\isaliteral{2C}{\isacharcomma}}q{\isadigit{2}}{\isadigit{1}}{\isadigit{4}}{\isaliteral{2C}{\isacharcomma}}q{\isadigit{3}}{\isadigit{1}}{\isadigit{1}}{\isaliteral{5D}{\isacharbrackright}}{\isaliteral{2C}{\isacharcomma}}a{\isadigit{1}}{\isaliteral{2C}{\isacharcomma}}p{\isadigit{1}}{\isadigit{3}}\isanewline
\ \ \ \ {\isaliteral{2C}{\isacharcomma}}loop\ {\isaliteral{5B}{\isacharbrackleft}}{\isaliteral{21}{\isacharbang}}a{\isadigit{4}}\ {\isaliteral{2B}{\isacharplus}}\ seq\ {\isaliteral{5B}{\isacharbrackleft}}a{\isadigit{4}}{\isaliteral{2C}{\isacharcomma}}{\isaliteral{28}{\isacharparenleft}}{\isaliteral{21}{\isacharbang}}a{\isadigit{2}}{\isaliteral{5C3C63646F743E}{\isasymcdot}}p{\isadigit{2}}{\isadigit{2}}{\isaliteral{29}{\isacharparenright}}\isaliteral{5C3C5E7375703E}{}\isactrlsup {\isaliteral{5C3C737461723E}{\isasymstar}}{\isaliteral{2C}{\isacharcomma}}a{\isadigit{2}}{\isaliteral{2C}{\isacharcomma}}{\isaliteral{21}{\isacharbang}}a{\isadigit{3}}{\isaliteral{2C}{\isacharcomma}}p{\isadigit{4}}{\isadigit{1}}{\isaliteral{2C}{\isacharcomma}}p{\isadigit{1}}{\isadigit{1}}{\isaliteral{5D}{\isacharbrackright}}\isanewline
{\isaliteral{2C}{\isacharcomma}}q{\isadigit{2}}{\isadigit{1}}{\isadigit{4}}{\isaliteral{2C}{\isacharcomma}}q{\isadigit{3}}{\isadigit{1}}{\isadigit{1}}{\isaliteral{2C}{\isacharcomma}}loop\ {\isaliteral{5B}{\isacharbrackleft}}{\isaliteral{21}{\isacharbang}}a{\isadigit{1}}{\isaliteral{2C}{\isacharcomma}}p{\isadigit{1}}{\isadigit{1}}{\isaliteral{2C}{\isacharcomma}}q{\isadigit{2}}{\isadigit{1}}{\isadigit{4}}{\isaliteral{2C}{\isacharcomma}}q{\isadigit{3}}{\isadigit{1}}{\isadigit{1}}{\isaliteral{5D}{\isacharbrackright}}{\isaliteral{2C}{\isacharcomma}}a{\isadigit{1}}{\isaliteral{2C}{\isacharcomma}}p{\isadigit{1}}{\isadigit{3}}{\isaliteral{5D}{\isacharbrackright}}\isanewline
\ \ \ \ {\isaliteral{2C}{\isacharcomma}}a{\isadigit{4}}{\isaliteral{2C}{\isacharcomma}}{\isaliteral{28}{\isacharparenleft}}{\isaliteral{21}{\isacharbang}}a{\isadigit{2}}{\isaliteral{5C3C63646F743E}{\isasymcdot}}p{\isadigit{2}}{\isadigit{2}}{\isaliteral{29}{\isacharparenright}}\isaliteral{5C3C5E7375703E}{}\isactrlsup {\isaliteral{5C3C737461723E}{\isasymstar}}{\isaliteral{2C}{\isacharcomma}}a{\isadigit{2}}{\isaliteral{2C}{\isacharcomma}}a{\isadigit{3}}{\isaliteral{2C}{\isacharcomma}}z{\isadigit{2}}{\isaliteral{2C}{\isacharcomma}}halt{\isaliteral{5D}{\isacharbrackright}}{\isaliteral{22}{\isachardoublequoteclose}}\isanewline
$=$\isanewline
seq {\isaliteral{5B}{\isacharbrackleft}}s{\isadigit{2}}{\isaliteral{2C}{\isacharcomma}}a{\isadigit{2}}{\isaliteral{2C}{\isacharcomma}}q{\isadigit{2}}{\isadigit{2}}{\isadigit{2}}{\isaliteral{2C}{\isacharcomma}}{\isaliteral{28}{\isacharparenleft}}seq\ {\isaliteral{5B}{\isacharbrackleft}}{\isaliteral{21}{\isacharbang}}a{\isadigit{2}}{\isaliteral{2C}{\isacharcomma}}r{\isadigit{2}}{\isadigit{2}}{\isaliteral{2C}{\isacharcomma}}a{\isadigit{2}}{\isaliteral{2C}{\isacharcomma}}q{\isadigit{2}}{\isadigit{2}}{\isadigit{2}}{\isaliteral{5D}{\isacharbrackright}}{\isaliteral{29}{\isacharparenright}}\isaliteral{5C3C5E7375703E}{}\isactrlsup {\isaliteral{5C3C737461723E}{\isasymstar}}{\isaliteral{2C}{\isacharcomma}}a{\isadigit{2}}{\isaliteral{2C}{\isacharcomma}}z{\isadigit{2}}{\isaliteral{2C}{\isacharcomma}}halt{\isaliteral{5D}{\isacharbrackright}}{\isaliteral{22}{\isachardoublequoteclose}}\isanewline
\end{isabellebody}

The proof consists of more than 450 lines of Isabelle code. After some
preparatory lemmas we essentially follow Angus and Kozen's proof line
by line. The proof is heavily based on our underlying KAT library
which contains about a hundred lemmas for dealing with the Kleene star
and combined reasoning about the interaction between actions and
tests. Typical properties are $(p+q)^\ast = p^\ast(q\cdot
p^\ast)^\ast$, $(pq)^\ast p = p(qp)^\ast$ or $bp = pc
\Longleftrightarrow bp!c = !bpc$. We have also programmed a number
tactics for manipulating large skat expressions. Most of these
implement commutations in lists of expressions modulo commutativity
conditions on atomic expressions which are inferred from the terms on
the fly. These tactics refine the simple commutativity check discussed
in Section \ref{sec:metatheorem}. The general strategy is to work with
the abbreviated terms using the rules of KAT, as far as possible and
descend to the level of SKAT by expanding abbreviations where
necessary, for instance to infer commutativity conditions using our
tactics.

As an example, if we abbreviate $1 := Var\ 2$ as $p$, and $3 := Var\ 4$
as $q$ the SKAT identity from section \ref{sec:metatheorem} reduces to
the KAT identity $pq = qp$.

Our previous experience in theorem proving with algebra shows that
proof automation is often very successful~\cite{self}. In this regard,
our present proof experience is slightly disappointing, as custom
tactics and low level proof techniques were needed to reconstruct
Angus and Kozen's proof. A further compression of their proof steps
does not seem feasible, a complete automation of their proof would
currently be out of reach. The main reason is that algebraic terms
generated by verification examples tend to be much longer than typical
textbook identities. An alternative proof strategy in which
commutativity conditions are not inferred by tactics on the fly, but
generated beforehand as abstract algebraic conditions might be
beneficial for automation. In addition, experience shows that
reasoning with inequalities in Kleene algebra is often superior to
equational reasoning (``Use inequalities whenever possible''
J.~H.~Conway), and it often leads to shorter terms. Theorem provers
however, find inequational reasoning more difficult.

% Flowcharts are constructed from the components shown in Figure
% \ref{fig:flow}.

% %% Could do this all using TikZ
% \begin{figure}[tbh]
%   \begin{center}
%     \begin{tikzpicture}
%       \node (c) {};
%       \node [test, below of=c] (test) {$\Phi$};
%       \node [below left of=test,node distance=1.1cm] (t) {};
%       \node [below right of=test, node distance=1.1cm] (f) {};
%       \path [line] (c) -- (test);
%       \path [line] (test) -| node[false] {F} (t);
%       \path [line] (test) -| node[true] {T} (f);
%     \end{tikzpicture}\\
%     If statement
%   \end{center}
%   \begin{minipage}[t]{0.49\columnwidth}
%     \begin{center}
%       \begin{tikzpicture}
%         \node [block] (start) {start};
%         \node [below of=start, node distance=0.8cm] (c) {};
%         \path [line] (start) -- (c);
%       \end{tikzpicture}\\
%       Start statement\\
%       \vspace{1.1cm}
%       \begin{tikzpicture}
%         \node (c) {};
%         \node [block, below of=c, node distance=0.8cm] (loop) {loop};
%         \path [line] (c) -- (loop);
%       \end{tikzpicture}\\
%       Loop statement
%     \end{center}
%   \end{minipage}
%   \begin{minipage}[t]{0.49\columnwidth}
%     \begin{center}
%       \begin{tikzpicture}
%         \node (c) {};
%         \node [block, below of=c, node distance=0.8cm] (halt) {halt};
%         \path [line] (c) -- (halt);
%       \end{tikzpicture}\\
%       Halt statement\\
%       \vspace{0.5cm}
%       \begin{tikzpicture}
%         \node (c1) {};
%         \node [block, below of=c1, node distance=0.8cm] (a) {$y_n = t$};
%         \node [below of=a, node distance=0.8cm] (c2) {};
%         \path [line] (c1) -- (a);
%         \path [line] (a) -- (c2);
%       \end{tikzpicture}\\
%       Assignment statement
%     \end{center}
%   \end{minipage}
%   \caption{Flowchart statements}
%   \label{fig:flow}
% \end{figure}

\begin{figure}[ptb]
\begin{minipage}[t]{0.49\columnwidth}
\begin{tikzpicture}[node distance=1cm,auto]
  \node [block] (start) {start};
  \node [block, below of=start] (a1) {$y := f(x)$};
  \node [test, below of=a1, node distance=1.6cm] (t1) {$P(y)$};
  \node [block, below left of=t1,node distance=1.7cm] (loop) {loop};
  \node [block, below right of=t1,node distance=1.7cm] (a2) {$y := g(y,y)$};
  \node [test, below of=a2, node distance=1.6cm] (t2) {$P(y)$};
  \node [block, below left of=t2,node distance=1.7cm] (a3) {$y = f(f(y))$};
  \node [block, below right of=t2,node distance=1.7cm] (a4) {$z := y$};
  \node [block, below of=a4] (halt) {halt};

  \path [line] (start) -- (a1);
  \path [line] (a1) -- (t1);
  \path [line] (t1) -| node[false] {F} (loop);
  \path [line] (t1) -| node[true] {T} (a2);
  \path [line] (a2) -- (t2);
  \path [line] (t2) -| node[false] {F} (a3);
  \path [line] (t2) -| node[true] {T} (a4);
  \path [line] (a4) -- (halt);
  \path [line] let \p1 = (a3.base),
                   \p2 = (t1.base)
               in (a3) -- (\x1,\y1 - 1cm) -- (\x1 - 2cm, \y1 - 1cm) --  (\x1 - 2cm, \y2 + 12mm) -- (\x2,\y2 + 12mm);
\end{tikzpicture}
\begin{center}
Scheme $S_{6E}$ \cite[p. 258]{Manna}
\end{center}
\end{minipage}
\begin{minipage}[t]{0.49\columnwidth}
\begin{tikzpicture}[node distance=1cm,auto]
  \node [block] (start) {start};
  \node [block, below of=start] (a1) {$y_1 := x$};
  \node [block, below of=a1] (a2) {$y_4 := f(y_1)$};
  \node [block, below of=a2] (a3) {$y_1 := f(y_1)$};
  \node [block, below of=a3] (a4) {$y_2 := g(y_1,y_4)$};
  \node [block, below of=a4] (a5) {$y_3 := g(y_1,y_1)$};
  \node [test, below of=a5, node distance=1.6cm] (t1) {$P(y_1)$};
  \node [block, below of=t1, node distance=1.6cm] (a6) {$y_1 := f(y_3)$};
  \node [test, below of=a6, node distance=1.6cm] (t2) {$P(y_4)$};
  \node [test, below of=t2, node distance=2.2cm] (t3) {$P(y_2)$};
  \node [block, below left of=t3,node distance=1.7cm] (a7) {$y_2 := f(y_2)$};
  \node [test, below right of=t3,node distance=1.7cm] (t4) {$P(y_3)$};
  \node [block, below of=t4, node distance=1.6cm] (a8) {$z := y_2$};
  \node [block, below of=a8] (halt) {halt};

  \path [line] (start) -- (a1);
  \path [line] (a1) -- (a2);
  \path [line] (a2) -- (a3);
  \path [line] (a3) -- (a4);
  \path [line] (a4) -- (a5);
  \path [line] (a5) -- (t1);
  \path [line] let \p1 = (t1.base), \p2 = (a3.base)
               in (t1) -| node{F}(\x1 - 1.5cm,\y1 + 1mm) -- (\x1 - 1.5cm,\y2 + 6.5mm) -- (\x2, \y2 + 6.5mm);
  \path [line] (t1) -- node{T} (a6);
  \path [line] (a6) -- (t2);
  \path [line] (t2) -- node{T} (t3);
  \path [line] let \p1 = (t2.base), \p2 = (a4.base)
               in (t2) -| node{F}(\x1 - 2.5cm,\y1 + 1mm) -- (\x1 - 2.5cm,\y2 + 6.5mm) -- (\x2, \y2 + 6.5mm);
  \path [line] (t3) -| node[true]{T} (t4);
  \path [line] (t3) -| node[false]{F} (a7);
  \path [line] (t4) -- node{T} (a8);
  \path [line] (a8) -- (halt);
  \path [line] let \p1 = (t4.base), \p2 = (a2.base)
               in (t4) -| node[false]{F}(\x1 + 1cm,\y1 + 1mm) -- (\x1 + 1cm,\y2 + 6.5mm) -- (\x2,\y2 + 6.5mm);
  \path [line] let \p1 = (a7.base), \p2 = (t3.base)
               in (a7) -- (\x1,\y1 - 1cm) -- (\x1 - 1.5cm,\y1 - 1cm) -- (\x1 - 1.5cm,\y2 + 13mm) -- (\x2,\y2 + 13mm);
\end{tikzpicture}
\begin{center}
Scheme $S_{6A}$ \cite[p. 254]{Manna}
\end{center}
\end{minipage}
\caption{Two equivalent program schemes}
\end{figure}

\section{Hoare Logic}

It is well known that PHL can be encoded in KAT, and other variants of
Kleene algebra such as modal Kleene algebras and Kleene modules. The
latter can be understood as algebraic versions of propositional
dynamic logic. A combination of these algebras with the assignment
rule and their application in formal program verification has so far
not been attempted, at least not in a mechanised theorem prover. We
have implemented a novel approach in which SKAT and Kleene modules are
combined. This allows us to separate tests conceptually from pre and
post-conditions of programs.

We define a \emph{Kleene module}~\cite{KM} as a structure
$(K,L,:)$ where $K$ is a Kleene algebra
$(K,+,\cdot,^\star,0,1)$, $L$ is a boolean algebra and $:$
is a mapping of type $L \times K \to L$ where
\begin{align*}
  m : (p + q) &= m : p \sqcup m : q,\\
  m : (p \cdot q) &= m : p : q,\\
  (m \sqcup n) : p &= m : p \sqcup n : p,\\
  m : 0 &= \bot,\\
  m : 1 &= m,\\
  (m \sqcup n) : p \le n &\longrightarrow m : p^\ast \le n.
\end{align*}
Instead of a Kleene algebra, we can use a KAT as the first component
of the module. To model the interaction between assertions as modelled
by $L$ and tests, we add the axiom
\begin{equation*}
  m : a = m \wedge (\top : a).
\end{equation*}

We have implemented modules over KAT in Isabelle by using locales,
based on that, we can define Hoare triples as usual.

\begin{isabellebody}
\isanewline
\isacommand{definition}\isamarkupfalse%
\ hoare{\isaliteral{5F}{\isacharunderscore}}triple\ {\isaliteral{3A}{\isacharcolon}}{\isaliteral{3A}{\isacharcolon}}\ {\isaliteral{22}{\isachardoublequoteopen}}{\isaliteral{27}{\isacharprime}}b\ {\isaliteral{5C3C52696768746172726F773E}{\isasymRightarrow}}\ {\isaliteral{27}{\isacharprime}}a\ {\isaliteral{5C3C52696768746172726F773E}{\isasymRightarrow}}\ {\isaliteral{27}{\isacharprime}}b\ {\isaliteral{5C3C52696768746172726F773E}{\isasymRightarrow}}\ bool{\isaliteral{22}{\isachardoublequoteclose}}\ {\isaliteral{28}{\isacharparenleft}}{\isaliteral{22}{\isachardoublequoteopen}}{\isaliteral{5F}{\isacharunderscore}}\ {\isaliteral{5C3C6C62726163653E}{\isasymlbrace}}\ {\isaliteral{5F}{\isacharunderscore}}\ {\isaliteral{5C3C7262726163653E}{\isasymrbrace}}\ {\isaliteral{5F}{\isacharunderscore}}{\isaliteral{22}{\isachardoublequoteclose}}\ {\isaliteral{5B}{\isacharbrackleft}}{\isadigit{5}}{\isadigit{4}}{\isaliteral{2C}{\isacharcomma}}{\isadigit{5}}{\isadigit{4}}{\isaliteral{2C}{\isacharcomma}}{\isadigit{5}}{\isadigit{4}}{\isaliteral{5D}{\isacharbrackright}}\ {\isadigit{5}}{\isadigit{3}}{\isaliteral{29}{\isacharparenright}}\ \isakeyword{where}\isanewline
\ \ {\isaliteral{22}{\isachardoublequoteopen}}P\ {\isaliteral{5C3C6C62726163653E}{\isasymlbrace}}\ p\ {\isaliteral{5C3C7262726163653E}{\isasymrbrace}}\ Q\ {\isaliteral{5C3C65717569763E}{\isasymequiv}}\ P\ {\isaliteral{5C3C436F6C6F6E3E}{\isasymColon}}\ p\ {\isaliteral{5C3C737173756273657465713E}{\isasymsqsubseteq}}\isaliteral{5C3C5E627375623E}{}\isactrlbsub A\isaliteral{5C3C5E657375623E}{}\isactrlesub \ Q{\isaliteral{22}{\isachardoublequoteclose}}\isanewline
\end{isabellebody}

As $:$ is a reserved symbol in Isabelle, we use $::$ for the scalar
product. The index $A$ refers to the boolean algebra of assertions and
the order $\sqsubseteq_A$ is the order on this boolean algebra. Using
this encoding, the inference rules of PHL (which does not include the
assignment rule) can now easily be derived as theorems of KAT
modules. As an example, we derive the while rule.

\begin{isabellebody}
\isanewline
\ \ \ \isacommand{lemma}\isamarkupfalse%
\ hoare{\isaliteral{5F}{\isacharunderscore}}while{\isaliteral{3A}{\isacharcolon}}\isanewline
\ \ \ \ \isakeyword{assumes}\ b{\isaliteral{5F}{\isacharunderscore}}test\ {\isaliteral{5B}{\isacharbrackleft}}intro{\isaliteral{5D}{\isacharbrackright}}{\isaliteral{3A}{\isacharcolon}}\ {\isaliteral{22}{\isachardoublequoteopen}}b\ {\isaliteral{5C3C696E3E}{\isasymin}}\ tests\ K{\isaliteral{22}{\isachardoublequoteclose}}\ \isakeyword{and}\ pc\ {\isaliteral{5B}{\isacharbrackleft}}intro{\isaliteral{5D}{\isacharbrackright}}{\isaliteral{3A}{\isacharcolon}}\ {\isaliteral{22}{\isachardoublequoteopen}}p\ {\isaliteral{5C3C696E3E}{\isasymin}}\ carrier\ K{\isaliteral{22}{\isachardoublequoteclose}}\isanewline
\ \ \ \ \isakeyword{and}\ Pc\ {\isaliteral{5B}{\isacharbrackleft}}intro{\isaliteral{5D}{\isacharbrackright}}{\isaliteral{3A}{\isacharcolon}}\ {\isaliteral{22}{\isachardoublequoteopen}}P\ {\isaliteral{5C3C696E3E}{\isasymin}}\ carrier\ A{\isaliteral{22}{\isachardoublequoteclose}}\isanewline
\ \ \ \ \isakeyword{and}\ loop{\isaliteral{5F}{\isacharunderscore}}condition\ {\isaliteral{5B}{\isacharbrackleft}}intro{\isaliteral{5D}{\isacharbrackright}}{\isaliteral{3A}{\isacharcolon}}\ {\isaliteral{22}{\isachardoublequoteopen}}P\ {\isaliteral{5C3C7371696E7465723E}{\isasymsqinter}}\ {\isaliteral{28}{\isacharparenleft}}{\isaliteral{5C3C746F703E}{\isasymtop}}\ {\isaliteral{5C3C436F6C6F6E3E}{\isasymColon}}\ b{\isaliteral{29}{\isacharparenright}}\ {\isaliteral{5C3C6C62726163653E}{\isasymlbrace}}p{\isaliteral{5C3C7262726163653E}{\isasymrbrace}}\ P{\isaliteral{22}{\isachardoublequoteclose}}\isanewline
\ \ \ \ \isakeyword{shows}\ {\isaliteral{22}{\isachardoublequoteopen}}P\ {\isaliteral{5C3C6C62726163653E}{\isasymlbrace}}\ {\isaliteral{28}{\isacharparenleft}}b{\isaliteral{5C3C63646F743E}{\isasymcdot}}p{\isaliteral{29}{\isacharparenright}}\isaliteral{5C3C5E7375703E}{}\isactrlsup {\isaliteral{5C3C737461723E}{\isasymstar}}{\isaliteral{5C3C63646F743E}{\isasymcdot}}{\isaliteral{21}{\isacharbang}}b\ {\isaliteral{5C3C7262726163653E}{\isasymrbrace}}\ P\ {\isaliteral{5C3C7371696E7465723E}{\isasymsqinter}}\ {\isaliteral{28}{\isacharparenleft}}{\isaliteral{5C3C746F703E}{\isasymtop}}\ {\isaliteral{5C3C436F6C6F6E3E}{\isasymColon}}\ {\isaliteral{21}{\isacharbang}}b{\isaliteral{29}{\isacharparenright}}{\isaliteral{22}{\isachardoublequoteclose}}\isanewline
%
\isadelimproof
\ \ %
\endisadelimproof
%
\isatagproof
\isacommand{proof}\isamarkupfalse%
\ {\isaliteral{28}{\isacharparenleft}}rule\ hoare{\isaliteral{5F}{\isacharunderscore}}composition{\isaliteral{5B}{\isacharbrackleft}}\isakeyword{where}\ {\isaliteral{3F}{\isacharquery}}Q\ {\isaliteral{3D}{\isacharequal}}\ P{\isaliteral{5D}{\isacharbrackright}}{\isaliteral{2C}{\isacharcomma}}\ auto{\isaliteral{29}{\isacharparenright}}\isanewline
\ \ \ \ \isacommand{show}\isamarkupfalse%
\ {\isaliteral{22}{\isachardoublequoteopen}}P\ {\isaliteral{5C3C6C62726163653E}{\isasymlbrace}}\ {\isaliteral{28}{\isacharparenleft}}b\ {\isaliteral{5C3C63646F743E}{\isasymcdot}}\ p{\isaliteral{29}{\isacharparenright}}\isaliteral{5C3C5E7375703E}{}\isactrlsup {\isaliteral{5C3C737461723E}{\isasymstar}}\ {\isaliteral{5C3C7262726163653E}{\isasymrbrace}}\ P{\isaliteral{22}{\isachardoublequoteclose}}\isanewline
\ \ \ \ \ \ \isacommand{by}\isamarkupfalse%
\ {\isaliteral{28}{\isacharparenleft}}blast\ intro{\isaliteral{3A}{\isacharcolon}}\ hoare{\isaliteral{5F}{\isacharunderscore}}star\ hoare{\isaliteral{5F}{\isacharunderscore}}composition{\isaliteral{5B}{\isacharbrackleft}}\isakeyword{where}\ {\isaliteral{3F}{\isacharquery}}Q\ {\isaliteral{3D}{\isacharequal}}\ {\isaliteral{22}{\isachardoublequoteopen}}P\ {\isaliteral{5C3C7371696E7465723E}{\isasymsqinter}}\ {\isaliteral{5C3C746F703E}{\isasymtop}}\ {\isaliteral{5C3C436F6C6F6E3E}{\isasymColon}}\ b{\isaliteral{22}{\isachardoublequoteclose}}{\isaliteral{5D}{\isacharbrackright}}{\isaliteral{29}{\isacharparenright}}\isanewline
\ \ \isacommand{qed}\isamarkupfalse%
%
\endisatagproof
\end{isabellebody}

To use this Hoare logic for verification, we need to model the
assignment axiom and the statespace of the program. The first can be
obtained by specialising KAT to SKAT, the second by explicitly
modelling the boolean algebra of states. As usual, states of a program
are represented as functions from variables to values. Assertions
correspond to sets of states. Hence the boolean algebra is a powerset
algebra over states. Similar implementations are available in theorem
provers such as Isabelle, HOL and Coq, but not as instantiations of
more general algebraic structures. SKAT expressions, in particular,
assignment statements are translated in Gordon style into forward
predicate transformers. These map assertions to assertions.

\begin{isabellebody}
\isanewline
\isacommand{fun}\isamarkupfalse%
\ eval{\isaliteral{5F}{\isacharunderscore}}skat{\isaliteral{5F}{\isacharunderscore}}expr\ {\isaliteral{3A}{\isacharcolon}}{\isaliteral{3A}{\isacharcolon}}\isanewline
\ \ {\isaliteral{22}{\isachardoublequoteopen}}{\isaliteral{28}{\isacharparenleft}}{\isaliteral{27}{\isacharprime}}a{\isaliteral{3A}{\isacharcolon}}{\isaliteral{3A}{\isacharcolon}}ranked{\isaliteral{5F}{\isacharunderscore}}alphabet{\isaliteral{2C}{\isacharcomma}}\ {\isaliteral{27}{\isacharprime}}b{\isaliteral{29}{\isacharparenright}}\ interp\ {\isaliteral{5C3C52696768746172726F773E}{\isasymRightarrow}}\ {\isaliteral{27}{\isacharprime}}a\ skat{\isaliteral{5F}{\isacharunderscore}}expr\ {\isaliteral{5C3C52696768746172726F773E}{\isasymRightarrow}}\ {\isaliteral{27}{\isacharprime}}b\ mems\ {\isaliteral{5C3C52696768746172726F773E}{\isasymRightarrow}}\ {\isaliteral{27}{\isacharprime}}b\ mems{\isaliteral{22}{\isachardoublequoteclose}}\isanewline
\ \ \isakeyword{where}\isanewline
\ \ {\isaliteral{22}{\isachardoublequoteopen}}eval{\isaliteral{5F}{\isacharunderscore}}skat{\isaliteral{5F}{\isacharunderscore}}expr\ D\ {\isaliteral{28}{\isacharparenleft}}SKLeaf\ x\ s{\isaliteral{29}{\isacharparenright}}\ {\isaliteral{5C3C44656C74613E}{\isasymDelta}}\ {\isaliteral{3D}{\isacharequal}}\ assigns\ D\ x\ s\ {\isaliteral{5C3C44656C74613E}{\isasymDelta}}{\isaliteral{22}{\isachardoublequoteclose}}\isanewline
{\isaliteral{7C}{\isacharbar}}\ {\isaliteral{22}{\isachardoublequoteopen}}eval{\isaliteral{5F}{\isacharunderscore}}skat{\isaliteral{5F}{\isacharunderscore}}expr\ D\ {\isaliteral{28}{\isacharparenleft}}SKBool\ P{\isaliteral{29}{\isacharparenright}}\ {\isaliteral{5C3C44656C74613E}{\isasymDelta}}\ {\isaliteral{3D}{\isacharequal}}\ filter{\isaliteral{5F}{\isacharunderscore}}set\ {\isaliteral{28}{\isacharparenleft}}eval{\isaliteral{5F}{\isacharunderscore}}bexpr\ D\ P{\isaliteral{29}{\isacharparenright}}\ {\isaliteral{5C3C44656C74613E}{\isasymDelta}}{\isaliteral{22}{\isachardoublequoteclose}}\isanewline
{\isaliteral{7C}{\isacharbar}}\ {\isaliteral{22}{\isachardoublequoteopen}}eval{\isaliteral{5F}{\isacharunderscore}}skat{\isaliteral{5F}{\isacharunderscore}}expr\ D\ {\isaliteral{28}{\isacharparenleft}}s\ {\isaliteral{3A}{\isacharcolon}}{\isaliteral{5C3C6F646F743E}{\isasymodot}}{\isaliteral{3A}{\isacharcolon}}\ t{\isaliteral{29}{\isacharparenright}}\ {\isaliteral{5C3C44656C74613E}{\isasymDelta}}\ {\isaliteral{3D}{\isacharequal}}\ eval{\isaliteral{5F}{\isacharunderscore}}skat{\isaliteral{5F}{\isacharunderscore}}expr\ D\ t\ {\isaliteral{28}{\isacharparenleft}}eval{\isaliteral{5F}{\isacharunderscore}}skat{\isaliteral{5F}{\isacharunderscore}}expr\ D\ s\ {\isaliteral{5C3C44656C74613E}{\isasymDelta}}{\isaliteral{29}{\isacharparenright}}{\isaliteral{22}{\isachardoublequoteclose}}\isanewline
{\isaliteral{7C}{\isacharbar}}\ {\isaliteral{22}{\isachardoublequoteopen}}eval{\isaliteral{5F}{\isacharunderscore}}skat{\isaliteral{5F}{\isacharunderscore}}expr\ D\ {\isaliteral{28}{\isacharparenleft}}s\ {\isaliteral{3A}{\isacharcolon}}{\isaliteral{5C3C6F706C75733E}{\isasymoplus}}{\isaliteral{3A}{\isacharcolon}}\ t{\isaliteral{29}{\isacharparenright}}\ {\isaliteral{5C3C44656C74613E}{\isasymDelta}}\ {\isaliteral{3D}{\isacharequal}}\ eval{\isaliteral{5F}{\isacharunderscore}}skat{\isaliteral{5F}{\isacharunderscore}}expr\ D\ s\ {\isaliteral{5C3C44656C74613E}{\isasymDelta}}\ {\isaliteral{5C3C756E696F6E3E}{\isasymunion}}\ eval{\isaliteral{5F}{\isacharunderscore}}skat{\isaliteral{5F}{\isacharunderscore}}expr\ D\ t\ {\isaliteral{5C3C44656C74613E}{\isasymDelta}}{\isaliteral{22}{\isachardoublequoteclose}}\isanewline
{\isaliteral{7C}{\isacharbar}}\ {\isaliteral{22}{\isachardoublequoteopen}}eval{\isaliteral{5F}{\isacharunderscore}}skat{\isaliteral{5F}{\isacharunderscore}}expr\ D\ {\isaliteral{28}{\isacharparenleft}}SKStar\ s{\isaliteral{29}{\isacharparenright}}\ {\isaliteral{5C3C44656C74613E}{\isasymDelta}}\ {\isaliteral{3D}{\isacharequal}}\ {\isaliteral{28}{\isacharparenleft}}{\isaliteral{5C3C556E696F6E3E}{\isasymUnion}}n{\isaliteral{2E}{\isachardot}}\ iter\ n\ {\isaliteral{28}{\isacharparenleft}}eval{\isaliteral{5F}{\isacharunderscore}}skat{\isaliteral{5F}{\isacharunderscore}}expr\ D\ s{\isaliteral{29}{\isacharparenright}}\ {\isaliteral{5C3C44656C74613E}{\isasymDelta}}{\isaliteral{29}{\isacharparenright}}{\isaliteral{22}{\isachardoublequoteclose}}\isanewline
{\isaliteral{7C}{\isacharbar}}\ {\isaliteral{22}{\isachardoublequoteopen}}eval{\isaliteral{5F}{\isacharunderscore}}skat{\isaliteral{5F}{\isacharunderscore}}expr\ D\ SKOne\ {\isaliteral{5C3C44656C74613E}{\isasymDelta}}\ {\isaliteral{3D}{\isacharequal}}\ {\isaliteral{5C3C44656C74613E}{\isasymDelta}}{\isaliteral{22}{\isachardoublequoteclose}}\isanewline
{\isaliteral{7C}{\isacharbar}}\ {\isaliteral{22}{\isachardoublequoteopen}}eval{\isaliteral{5F}{\isacharunderscore}}skat{\isaliteral{5F}{\isacharunderscore}}expr\ D\ SKZero\ {\isaliteral{5C3C44656C74613E}{\isasymDelta}}\ {\isaliteral{3D}{\isacharequal}}\ {\isaliteral{7B}{\isacharbraceleft}}{\isaliteral{7D}{\isacharbraceright}}{\isaliteral{22}{\isachardoublequoteclose}}\isanewline
\end{isabellebody}
\begin{isabellebody}
\isanewline
\isacommand{theorem}\isamarkupfalse%
\ skat{\isaliteral{5F}{\isacharunderscore}}con{\isaliteral{5F}{\isacharunderscore}}eval{\isaliteral{3A}{\isacharcolon}}\ {\isaliteral{22}{\isachardoublequoteopen}}\isanewline
\ \ skat{\isaliteral{5F}{\isacharunderscore}}con\ s\ t\ {\isaliteral{5C3C4C6F6E6772696768746172726F773E}{\isasymLongrightarrow}}\ {\isaliteral{5C3C666F72616C6C3E}{\isasymforall}}{\isaliteral{5C3C44656C74613E}{\isasymDelta}}{\isaliteral{2E}{\isachardot}}\ eval{\isaliteral{5F}{\isacharunderscore}}skat{\isaliteral{5F}{\isacharunderscore}}expr\ D\ s\ {\isaliteral{5C3C44656C74613E}{\isasymDelta}}\ {\isaliteral{3D}{\isacharequal}}\ eval{\isaliteral{5F}{\isacharunderscore}}skat{\isaliteral{5F}{\isacharunderscore}}expr\ D\ t\ {\isaliteral{5C3C44656C74613E}{\isasymDelta}}{\isaliteral{22}{\isachardoublequoteclose}}\isanewline
\end{isabellebody}
\begin{isabellebody}
\isanewline
\isacommand{lift{\isaliteral{5F}{\isacharunderscore}}definition}\isamarkupfalse%
\ eval\ {\isaliteral{3A}{\isacharcolon}}{\isaliteral{3A}{\isacharcolon}}\isanewline
\ \ {\isaliteral{22}{\isachardoublequoteopen}}{\isaliteral{28}{\isacharparenleft}}{\isaliteral{27}{\isacharprime}}a{\isaliteral{3A}{\isacharcolon}}{\isaliteral{3A}{\isacharcolon}}ranked{\isaliteral{5F}{\isacharunderscore}}alphabet{\isaliteral{2C}{\isacharcomma}}\ {\isaliteral{27}{\isacharprime}}b{\isaliteral{29}{\isacharparenright}}\ interp\ {\isaliteral{5C3C52696768746172726F773E}{\isasymRightarrow}}\ {\isaliteral{27}{\isacharprime}}a\ skat\ {\isaliteral{5C3C52696768746172726F773E}{\isasymRightarrow}}\ {\isaliteral{27}{\isacharprime}}b\ mems\ {\isaliteral{5C3C52696768746172726F773E}{\isasymRightarrow}}\ {\isaliteral{27}{\isacharprime}}b\ mems{\isaliteral{22}{\isachardoublequoteclose}}\isanewline
\ \ \isakeyword{is}\ eval{\isaliteral{5F}{\isacharunderscore}}skat{\isaliteral{5F}{\isacharunderscore}}expr\isanewline
\end{isabellebody}

{
\small
\begin{verbatim}
structure HoareSimpRules = Named_Thms
  (val name = @{binding "hoare_simp"}
   val description = "Simplification rules for the hoare tactic")

fun hoare_step_tac ctxt n =
  rtac @{thm hoare_composition} n
  ORELSE (rtac @{thm hoare_assignment} n
    THEN TRY (rtac @{thm subset_refl} n))
  ORELSE (rtac @{thm hoare_while_inv} n
    THEN asm_full_simp_tac (simpset_of ctxt) 1)

val hoare_tac = Subgoal.FOCUS (fn {context, ...} =>
  REPEAT (hoare_step_tac context 1)
  THEN auto_tac (map_simpset
    (fn ss => ss addsimps HoareSimpRules.get context) context))
\end{verbatim}
}

% \begin{minipage}[T]{0.1\textwidth}
% {
% \begin{align*}
% &\text{\textbf{lemma} hoare\_plus:}\\
% &\hspace{2ex}\text{\textbf{assumes} pc [intro]: p $\in$ $K$ \textbf{and} qc [intro]: q $\in$ $K$}\\
% &\hspace{2ex}\text{\textbf{and} mc [intro]: m $\in$ $A$ \textbf{and} nc [intro]: n $\in$ $A$}\\
% &\hspace{2ex}\text{\textbf{and} left\_branch: $\{m\} p \{n\}$}\\
% &\hspace{2ex}\text{\textbf{and} right\_branch: $\{m\} q \{n\}$}\\
% &\hspace{2ex}\text{\textbf{shows} $\{m\} p + q \{n\}$}\\
% &\text{\textbf{proof} -}\\
% &\hspace{2ex}\text{\textbf{have} $m : p \le n$ \textbf{and} $m : q \le n$}\\
% &\hspace{4ex}\text{\textbf{by} (insert left\_branch right\_branch) (simp add: hoare\_triple\_def)+}\\
% &\hspace{2ex}\text{\textbf{hence} $m : (p + q) \le n$}\\
% &\hspace{4ex}\text{\textbf{by} (simp add: mod\_plus[OF pc qc mc], subst $\mathcal{A}$.bin\_lub\_var, auto)}\\
% &\hspace{2ex}\text{\textbf{thus} ?thesis}\\
% &\hspace{4ex}\text{\textbf{by} (simp add: hoare\_triple\_def)}\\
% &\text{\textbf{qed}}
% \end{align*}
% }
% \end{minipage}

% \begin{minipage}[T]{0.1\textwidth}
% {
% \begin{align*}
% &\text{\textbf{lemma} \textit{hoare\_plus}:}\\
% &\hspace{2ex}\text{\textbf{assumes} \textit{pc} [intro]: \texttt{"}p $\in$ $K$\texttt{"} \textbf{and} \textit{qc} [intro]: \texttt{"}q $\in$ $K$\texttt{"}}\\
% &\hspace{2ex}\text{\textbf{and} \textit{Pc} [intro]: \texttt{"}P $\in$ $A$\texttt{"} \textbf{and} \textit{Qc} [intro]: \texttt{"}Q $\in$ $A$\texttt{"}}\\
% &\hspace{2ex}\text{\textbf{and} \textit{then\_branch}: \texttt{"}P $\llbrace$ p $\rrbrace$ Q\texttt{"}}\\
% &\hspace{2ex}\text{\textbf{and} \textit{else\_branch}: \texttt{"}P $\llbrace$ q $\rrbrace$ Q\texttt{"}}\\
% &\hspace{2ex}\text{\textbf{shows} \texttt{"}P $\llbrace$ p + q $\rrbrace$ Q\texttt{"}}\\
% &\text{\textbf{proof} -}\\
% &\hspace{2ex}\text{\textbf{have} \texttt{"}P $:$ p $\le$ Q\texttt{"} \textbf{and} \texttt{"}P $:$ q $\le$ Q\texttt{"}}\\
% &\hspace{4ex}\text{\textbf{by} (insert \textit{then\_branch else\_branch}) (simp add: \textit{hoare\_triple\_def})+}\\
% &\hspace{2ex}\text{\textbf{hence} \texttt{"}P $:$ (p + q) $\le$ Q\texttt{"}}\\
% &\hspace{4ex}\text{\textbf{by} (simp add: \textit{mod\_plus}[OF \textit{pc} \textit{qc} \textit{Pc}], subst \textit{A.bin\_lub\_var}, auto)}\\
% &\hspace{2ex}\text{\textbf{thus} ?thesis}\\
% &\hspace{4ex}\text{\textbf{by} (simp add: \textit{hoare\_triple\_def})}\\
% &\text{\textbf{qed}}
% \end{align*}
% }
% \end{minipage}

\section{Examples}

Euclid's algorithm for finding the greatest common divisor:

\begin{minipage}[T]{0.1\textwidth}
{
\begin{align*}
&\text{\textbf{lemma} \textit{euclids\_algorithm}:}\\
&\hspace{2ex}\text{\texttt{"}\{mem. mem 0 = x $\wedge$ mem 1 = y\} $\llbrace$}\\
&\hspace{2ex}\text{WHILE !(pred (EQ (Var 1) (NAT 0)))}\\
&\hspace{2ex}\text{INVARIANT \{mem. gcd (mem 0) (mem 1) = gcd x y\}}\\
&\hspace{2ex}\text{DO}\\
&\hspace{4ex}\text{2 := Var 1;}\\
&\hspace{4ex}\text{1 := MOD (Var 0) (Var 1);}\\
&\hspace{4ex}\text{0 := Var 2}\\
&\hspace{2ex}\text{WEND}\\
&\hspace{2ex}\text{$\rrbrace$ \{mem. mem 0 = gcd x y\}\texttt{"}}\\
&\hspace{2ex}\text{\textbf{by} hoare\_auto (metis \textit{gcd\_red\_nat})}
\end{align*}
}
\end{minipage}

Multiplication by repeated addition

\begin{minipage}[T]{0.1\textwidth}
{
\begin{align*}
&\text{\textbf{lemma} \textit{repeated\_addition}:}\\
&\hspace{2ex}\text{\texttt{"}\{mem. mem 0 = x $\wedge$ mem 1 = y\} $\llbrace$}\\
&\hspace{2ex}\text{2 := NAT 0;}\\
&\hspace{2ex}\text{3 := NAT 0;}\\
&\hspace{2ex}\text{WHILE !(pred (EQ (Var 2) (Var 0)))}\\
&\hspace{2ex}\text{INVARIANT \{mem. mem 3 = mem 2 * mem 1 $\wedge$ mem 0 = x $\wedge$ mem 1 = y\}}\\
&\hspace{2ex}\text{DO}\\
&\hspace{4ex}\text{3 := PLUS (Var 3) (Var 1);}\\
&\hspace{4ex}\text{2 := PLUS (Var 2) (NAT 1)}\\
&\hspace{2ex}\text{WEND}\\
&\hspace{2ex}\text{$\rrbrace$ \{mem. mem 3 = (x * y)\}\texttt{"}}\\
&\hspace{2ex}\text{\textbf{by} hoare\_auto smt}
\end{align*}
}
\end{minipage}

Factorial algorithm

\begin{minipage}[T]{0.1\textwidth}
{
\begin{align*}
&\text{\textbf{lemma} \textit{factorial}:}\\
&\hspace{2ex}\text{\texttt{"}\{mem. mem 0 = x\} $\llbrace$}\\
&\hspace{2ex}\text{1 := NAT 1;}\\
&\hspace{2ex}\text{WHILE !(pred (EQ (Var 0) (NAT 0)))}\\
&\hspace{2ex}\text{INVARIANT \{mem. fact x = mem 1 * fact (mem 0)\}}\\
&\hspace{2ex}\text{DO}\\
&\hspace{4ex}\text{1 := MULT (Var 1) (Var 0);}\\
&\hspace{4ex}\text{0 := MINUS (Var 0) (NAT 1)}\\
&\hspace{2ex}\text{WEND}\\
&\hspace{2ex}\text{$\rrbrace$ \{mem. mem 1 = fact x\}\texttt{"}}\\
&\hspace{2ex}\text{\textbf{by} hoare\_auto (metis \textit{fact\_reduce\_nat})}
\end{align*}
}
\end{minipage}

\section{Conclusion}

\bibliographystyle{plain}
\bibliography{paper}

\end{document}
